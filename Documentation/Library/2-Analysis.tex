% !TeX spellcheck = de_CH
\chapter{Analyse}
Das Ziel der Software ist, den User in der Terminfindung, Protokollierung und Partnerfindung optimal zu unterst�tzen. Dieser Teil der Dokumentation dient zur Findung exakter Anforderungen an die Software die den User optimal unterst�tzen. Zus�tzlich wird Anfangs untersucht wie das Marktumfeld rund um die geplante Applikation aussieht um einen m�glichen Erfolg einer solchen Applikation zu sch�tzen, sowie m�gliche Synergieeffekte zu identifizieren. 

\section{Marktumfeld}
Um m�gliche Synergien oder Wettbewerber zu identifizieren, m�ssen zuerst die geplanten Basis Funktionalit�ten aufgelistet werden:
\begin{itemize}
	 \itemsep-0.5em
	 \item Vereinfachung zur Identifizierung von Partnern
	 \item Vereinfachung zur Terminvereinbarung
	 \item Vereinfachung eines Amateur-Liga Management
	 \item Vereinfachung zu Protokollierung eines Spiels
	 
\end{itemize}

\subsection{Identifizierung von Partnern}
\textbf{GlobalTennisNetwork.com}
\textbf{Spontacts}
\textbf{sport42.com}
\textbf{sportpartner.com}


\subsection{Terminvereinbarung}
 \textbf{Doodle}
 

\subsection{Amateur-Liga Management}

\subsection{Protokollierung eines Spiels}

\section{Benutzergruppen}

\subsection{Gelegenheitsspieler}

\subsection{Regelm�ssige Spieler}

\subsection{Clubmitglieder}

\section{Use Cases}

\subsection{Identifizierung von Partnern}

\subsection{Terminvereinbarung}
\subsubsection{Gelegenheitspiele}

\subsubsection{Regelm�ssige Spiele}

\subsection{Spiel Protokollieren}

\subsection{Liga Management}
\section{Anforderungsanalyse}
