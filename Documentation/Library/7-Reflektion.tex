% !TeX spellcheck = de_CH
\chapter{Reflektion}