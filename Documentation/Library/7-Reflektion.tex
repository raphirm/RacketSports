% !TeX spellcheck = de_CH
\chapter{Reflektion}
Alles in allem war das Projekt eine spannende Erfahrung, welche st�rken und schw�chen in meiner Arbeit aufzeigte. Sch�tzungsweise wurden mehr als die 120 Stunden investiert durch die Komplexit�t der verschiedenen Workflows. Vor allem Ligamanagement der Matchworkflow wurde in komplexerer Algorithmus als erwartet. 

\section{Analyse und Dokumentation}
Nach einer Basis-Markt Analyse war noch eine Umfrage in einem nahen Racketsportzentrum geplant. Das Managment dieser Anlage wollte daf�r keine Unterst�tzung bieten, weshalb die Umfrage abgebrochen wurde. Trotzdem denke ich, wurde die Marktanalyse sehr gut erfasst und die verschiedenen Benutzergruppen gut erfasst. 

Die Anschliessenden Benutzerf�lle sowie Anforderungen wurden anfangs sehr rudiment�r und nicht standartisiert niedergeschrieben. Ich habe mich mehr auf die Implementation gest�tzt. Der Betreuer --- Herr Reiser --- hat mich freundlicherweise darauf hingewiesen, dass die Anforderungsanalyse in der Bewertung durchaus wichtig ist, weshalb ich diese nun auch nach Unterlagen durchgef�hrt habe. 

Der Aufwand zum dokumentieren war schlussendlich mehr als erwartet, weshalb ich am Schluss der Projektes noch etwas knapp war. 
\section{Konzeption}
Die Konzeption der Applikation war von Anfang an nicht der Hauptfokus des Projektes. Trotzdem wollte ich eine nachvollziehbare Argumentation f�r die verwendeten Patterns geben. Ich denke es gen�gt zwei Stacks miteinander zu vergleichen, da die Implementation schlussendlich im Fokus stand.

\section{Implementation}
Die Implementation hat sehr viel Zeit gekostet. Ein betr�chtlicher Teil des Aufwandes musste ich betreiben um �berhaupt den Stack, JavaScript und die Eigenheiten kennenzulernen. Anschliessend waren die Komplexen Algorithmen und verwendeten Libraries eine Herausforderung f�r sich zu implementieren. 

\section{Zeitmanagement}
Ich hatte am Anfang etwas wenig f�r das Projekt gearbeitet, weshalb ich dann bei der Implementation etwas in Zeitverzug kam. 