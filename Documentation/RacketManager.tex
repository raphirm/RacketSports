\documentclass[12pt,a4paper,openany,ngerman,plainfootsepline,plainheadsepline]{scrbook}
% Change "article" to "report" to get rid of page number on title page
\usepackage{amsmath,amsfonts,amsthm,amssymb}
\usepackage{setspace}
\usepackage{Tabbing}
\usepackage{lastpage}
\usepackage[backend=biber,citestyle=verbose-note]{biblatex}
\addbibresource{bib/verzeichnis1.bib}
\usepackage{here} 
\usepackage{tocbasic}
\usepackage{color}
\usepackage{listings}
\usepackage{pifont}
\usepackage{enumitem}
\usepackage{placeins}
\usepackage{lscape}
\usepackage{colortbl}
\usepackage{tabu}
\usepackage{longtable}
\usepackage{cals}
\usepackage{listings}




\definecolor{testblau}{rgb}{0.63, 0.79, 0.95}
\definecolor{airforceblue}{rgb}{0.36, 0.54, 0.66}
\definecolor{bluegray}{rgb}{0.4, 0.6, 0.8}
\definecolor{mygreen}{rgb}{0,0.6,0}
\definecolor{mygray}{rgb}{0.5,0.5,0.5}
\definecolor{mymauve}{rgb}{0.58,0,0.82}

\lstset{ %
	backgroundcolor=\color{white},   % choose the background color; you must add \usepackage{color} or \usepackage{xcolor}
	basicstyle=\footnotesize,        % the size of the fonts that are used for the code
	breakatwhitespace=false,         % sets if automatic breaks should only happen at whitespace
	breaklines=true,                 % sets automatic line breaking
	captionpos=b,                    % sets the caption-position to bottom
	commentstyle=\color{mygreen},    % comment style
	deletekeywords={...},            % if you want to delete keywords from the given language
	escapeinside={\%*}{*)},          % if you want to add LaTeX within your code
	extendedchars=true,              % lets you use non-ASCII characters; for 8-bits encodings only, does not work with UTF-8
	frame=single,
	keepspaces=true,                 % keeps spaces in text, useful for keeping indentation of code (possibly needs columns=flexible)
	keywordstyle=\color{blue},       % keyword style
	language=Octave,                 % the language of the code
	otherkeywords={*,...},            % if you want to add more keywords to the set
	numbers=left,                    % where to put the line-numbers; possible values are (none, left, right)
	numbersep=5pt,                   % how far the line-numbers are from the code
	numberstyle=\tiny\color{mygray}, % the style that is used for the line-numbers
	rulecolor=\color{black},         % if not set, the frame-color may be changed on line-breaks within not-black text (e.g. comments (green here))
	showspaces=false,                % show spaces everywhere adding particular underscores; it overrides 'showstringspaces'
	showstringspaces=false,          % underline spaces within strings only
	showtabs=false,                  % show tabs within strings adding particular underscores
	stepnumber=1,                    % the step between two line-numbers. If it's 1, each line will be numbered
	stringstyle=\color{mymauve},     % string literal style
	tabsize=2,	                   % sets default tabsize to 2 spaces
	title=\lstname                   % show the filename of files included with \lstinputlisting; also try caption instead of title
}




\usepackage[automark,						%Automatische Kopfzeile
						%headtopline,				%Linie �ber dem Seitenkopf
						%plainheadtopline,	%Plain, Linie �ber dem Seitenkopf
						headsepline,				%Linie zwischen Kopf und Textk�rper
						%plainheadsepline,	%Plain, Linie zwischen Kopf und Textk�rper
						footsepline,				%Linie zwischen Textk�rper und Fu�
						plainfootsepline,   %Plain, Linie zwischen Textk�rper und Fu�
						%footbotline,				%Linie unter dem Fu�
						%plainfootbotline   %Plain, Linie unter dem Fu�
						]{scrpage2}
\usepackage{graphicx,wrapfig}
\usepackage[ansinew]{inputenc}
\usepackage[ngerman]{babel}

\usepackage[hidelinks]{hyperref}


% In case you need to adjust margins:
\topmargin=-0.45in      %
\evensidemargin=0in     %
\oddsidemargin=0in      %
\textwidth=6.5in        %
\textheight=9.0in       %
\headsep=0.25in         %
\setcounter{secnumdepth}{3}
\setcounter{tocdepth}{3}
%\pdfliteral direct {/Interpolate true}
%\special {pdf:direct: /Interpolate true }
% Homework Specific Information
\newcommand{\hmwkTitle}{Racketsports Manager}
\newcommand{\hmwkClass}{Semesterarbeit ZHAW}
\newcommand{\hmwkAuthorName}{Raphael Marques}
\newcommand{\hmwkTeacherName}{Michael Reiser}


                                   %
\clearscrplain		
%Alte Plain-Formatierung entfernen
\cehead{\headmark}    % Chaper auf geraden Seiten (links) in Kopfzeile
\cohead{\headmark}
\rehead{\includegraphics[width=25pt]{Graphics/zhaw.jpg}}    % Section auf ungeraden Seiten (rechts) in Kopfzeile
\rohead{\includegraphics[width=25pt]{Graphics/zhaw.jpg}} 
\lehead{\hmwkTitle}    % Section auf ungeraden Seiten (rechts) in Kopfzeile
\lohead{\hmwkTitle} 
\refoot{\hmwkAuthorName}    % Chaper auf geraden Seiten (links) in Kopfzeile
\rofoot{\hmwkAuthorName}
\lofoot{Page\ \thepage\ of\ \pageref{LastPage}}    % Chaper auf geraden Seiten (links) in Kopfzeile
\lefoot{Page\ \thepage\ of\ \pageref{LastPage}}
\setheadsepline{0.4pt}   
\setfootsepline{0.4pt}                                  %
\pagestyle{scrheadings}
\automark[section]{chapter}
 % Seitenstil aktivieren
\renewcommand{\chapterpagestyle}{scrheadings}
% This is used to trace down (pin point) problems
% in latexing a document:
%\tracingall
\pdfpxdimen=1in

\divide\pdfpxdimen by 800

%%%%%%%%%%%%%%%%%%%%%%%%%%%%%%%%%%%%%%%%%%%%%%%%%%%%%%%%%%%%%
% Make title
\title{\vspace{1in}\textmd{\textbf{\ \hmwkTitle}}
\\\normalsize \vspace{0.1in} \large{\hmwkClass} \vspace{0.5in}
\\\includegraphics[width=300pt]{Graphics/title.png}
\vspace{0.1in}\large{\textit{}}\vspace{0.2in}
\author{\textbf{Autor: \hmwkAuthorName}
\\\textbf{ Betreuer: \hmwkTeacherName}}}
%%%%%%%%%%%%%%%%%%%%%%%%%%%%%%%%%%%%%%%%%%%%%%%%%%%%%%%%%%%%%



\begin{document}
\nocite{*}
\begin{spacing}{1.1}

\maketitle
\newpage
% Uncomment the \tableofcontents and \newpage lines to get a Contents page
% Uncomment the \setcounter line as well if you do NOT want subsections
%       listed in Contents
%\setcounter{tocdepth}{1}
%\tableofcontents
%\newpage

% When problems are long, it may be desirable to put a \newpage or a
% \clearpage before each homeworkProblem envirosnment

\clearpage
\begin{normalsize}
\setcounter{tocdepth}{2}
\tableofcontents
% !TeX spellcheck = de_CH


\chapter{Inhalt der Arbeit}

\section{Motivation}


\section{Aufgabenstellung}


\section{Vorgehen}


\section{Zielsetzung der Arbeit}

% !TeX spellcheck = de_CH
\newpage

\subsection{Grobplanung}
\begin{figure}[ht]
	\centering
	\includegraphics[width=1\textwidth, ]{Graphics/Grobplanung.png}
	\caption{Grobplanung f�r Applikation}
	\label{fig1}
\end{figure}
\FloatBarrier
\subsection{Termine}
\begin{itemize}
\item Kick-Off: 13. M�rz 2015
\item Design-Review: 22. Juli 2015
\item Abgabe: 20. August 2015
\item Pr�sentation: 30 September 2015
\end{itemize}




% !TeX spellcheck = de_CH
\chapter{Analyse}
Das Ziel der Software ist, den User in der Terminfindung, Protokollierung und Partnerfindung optimal zu unterst�tzen. Dieser Teil der Dokumentation dient zur Findung exakter Anforderungen an die Software die den User optimal unterst�tzen. Zus�tzlich wird Anfangs untersucht wie das Marktumfeld rund um die geplante Applikation aussieht um einen m�glichen Erfolg einer solchen Applikation zu sch�tzen, sowie m�gliche Synergieeffekte zu identifizieren. 

\section{Marktumfeld}
Um m�gliche Synergien oder Wettbewerber zu identifizieren, m�ssen zuerst die geplanten Basis Funktionalit�ten aufgelistet werden:
\begin{itemize}
	 \itemsep-0.5em
	 \item Vereinfachung zur Identifizierung von Partnern
	 \item Vereinfachung zur Terminvereinbarung
	 \item Vereinfachung eines Amateur-Liga Management
	 \item Vereinfachung zu Protokollierung eines Spiels

\subsection{Identifizierung von Partnern}

\subsection{Terminvereinbarung}

\subsection{Amateur-Liga Management}

\subsection{Protokollierung eines Spiels}

\section{Benutzergruppen}

\section{Use Cases}

\section{Anforderungsanalyse}

% !TeX spellcheck = de_DE
\chapter{Konzeption}
\begin{figure}[ht]
	\centering
	\includegraphics[width=0.5\textwidth]{Graphics/KonzeptApp.png}
	\caption{Grobkonzept f�r Applikation}
	\label{fig1}
\end{figure}
\FloatBarrier
Die Applikation besteht aus drei Teilen. Einem Webserver, der eine API und statische Clientfiles zur Verf�gung stellt, der Client im Browser, welcher die API konsumiert sowie eine Android Applikation welche die Website l�dt. 
 
\section{Technologiestack}
Wie in dem Grobkonzept beschrieben, wird f�r die Applikation einen Technologiestack gebraucht, welcher  eine skalierbare API sowie eine gute Integration der API mit einer Browser Frontend Anwendung bietet. Folgende Anforderungen werden an den Technologiestack gestellt:
\begin{itemize}
	\itemsep-0.5em
	\item Skalierbare REST API
	\item Einfacher und schneller Umgang mit AJAX
	\item Gute Integration zwischen API und HTML/JS Client
	\item Responsive Design, Integration mit OAUTH f�r 3rd Party Authentisierung
	\item Persistance Layer (Datenbankunterst�tzung)
\end{itemize} 

\subsection{MVC Frameworks}
Komplexe Applikationen werden vorzugsweise mit MVC Frameworks erstellt. �ber MVC APIs sind Routing,  Logik sowie Pr�sentationsschicht gut voneinander Abstrahiert. Es ist gut m�glich, f�r Views mit verschiedenen Daten �ber ein Model zu versehen, oder auch ein Logisches Routing f�r eine API zu entwickeln. Folgende Graphik zeigt ein MVC Konzept. Der Client sendet einen Request zu dem Server und wird vom Routing zur Logik im System weitergeleitet. Die Logik findet das richtige Model sowie die dazugeh�rige View.  Die View wird mit dem Model gerendert und es entsteht eine Antwort, welche dem Client zur�ckgesendet wird. In einer REST API ist die View JSON. Das Model wird in JSON umgewandelt und versendet. 

Folgendes Code-Beispiel - die Funktion list() -  zeigt, wie alle Courts aus dem Persistance Layer selektiert werden und per JSON zum Client gesendet werden. Dies ist ein API Endpunkt zur auflistung von Courts (http://webserver/courts). Gut zu sehen ist, dass die jsonp() Funktion als Renderer gebraucht wird anstatt eine Standard-View. 
\begin{lstlisting}
exports.list = function(req, res) { 
	Court.find().sort('-created').populate('user')
	 .exec(function(err, courts) {
		if (err) {
			return res.status(400).send({
				message: 
				 errorHandler.getErrorMessage(err)
			});
		} else {
		res.jsonp(courts);
		}
	});
};
\end{lstlisting}

Mit einem MVC Framework auf der Server-Seite kann man so gut abstrahierte und ausbauf�hige - skalierbare - APIs erstellen. Diese APIs m�ssen nun jedoch vom Client Browser verarbeitet werden k�nnen. Folgende M�glichkeiten bieten sich an:
\begin{itemize}
	\item Parallel zu der REST API werden Renderer gebaut, welche das Model mit einer View in eine - f�r den Client statische - Website rendern. Der Client verf�gt hier ausschliesslich Logik um die verschiedenen Webseiten abzurufen. 
	\item Ein Website-Skelett mit Logik wird beim ersten Aufruf an den Client verschickt. Der Client bezieht nun Daten aus der REST API und reichert die schon vorhandenen Views mit den Objekten - gesendet �ber AJAX - selber an. 
\end{itemize}
Eine parallele Implementierung zur Rest API geht entgegen dem Basis Konzept, dass alle Applikationen so gut wie m�glich von der REST API profitieren. Zus�tzlich w�rde bei einer parallelen Implementierung jeder Klick in einer Aktualisierung der Applikation resultieren. Dies ist unerw�nscht, da sich die Website nicht schnell und intuitiv anf�hlt. Man hat bei jedem Klick eine Downtime, da viel Daten �bertragen werden m�ssen, und der Browser den DOM jedes mal neu Aufbauen muss. Bei der zweiten Option wird die Website nur einmal heruntergeladen. Der DOM wird nach dem Download aufgebaut und von der Logik ver�ndert. Klicks l�sen einen viel geringeren Aufwand von Server bis Client aus und somit ist die Downtime viel kleiner. Die Applikation f�hlt sich schneller und intuitiver an. 

Wie bei dem Server, kann man auch bei der Applikation ein MVC Pattern implementieren. Ein Routing definiert, bei welchr URL welche View aufgerufen wird. Bei dem Aufruf einer View ist ein Controller hinterlegt, welcher bei der API das Model und Objekt besorgt. Die View rendert die vom Controller generierten Daten

\glqq Figure Angular JS \grqq 

Server und Client MVCs k�nnen so miteinander Kombiniert werden und es entsteht eine skalierbare und wartbare Applikationsumgebung.


\subsubsection{Server MVCs}
MVCs f�r den Server gibt es verschiedene:
\begin{itemize}
	\item Spring Framework - Java
	\item ExpressJS - JSON
	\item Rails - Ruby
\end{itemize}

\subsubsection{Client MVCs}
MVC f�r den Server sind ausschliesslich in Javascript geschrieben:
\begin{itemize}
	\item AngularJS
	\item Backbone.js
	\item Ember.js
	\end{itemize}

\subsubsection{Stacks}
Eine Konfiguration des Client- sowie Server MVCs, damit beide gut miteinander Funktionieren ist zus�tzlich wichtig. In der Evaluation wird somit folgende Konfigurationen abgewogen:
\begin{itemize}
	\item Spring Framework 
	\item Express Framework 
\end{itemize}

\subsubsection{Anforderungen an Stacks}
Die Anforderungen an einen Frameworkstack gehen aus den nicht funktionalen Anforderungen heraus. Einige Anforderungen, welche sich auf die User Experience beziehen (Benutzbarkeit) sind f�r die Anforderungen an den Stack nicht relevant. 
\begin{center}
	\tabulinesep = 1mm
	\begin{longtabu} to \linewidth [m]{X[1, m , l]|[2pt]p{2.95cm}|[2pt]p{2.5cm}|[2pt]}
		\arrayrulecolor{white}
		
		\tabucline[2pt]{-}
		\textcolor{white}{\textbf{\cellcolor{airforceblue}Anforderung}}  &  
		\textcolor{white}{\textbf{\cellcolor{airforceblue}Notwendigkeit}}&
		\textcolor{white}{\textbf{\cellcolor{airforceblue}Kritikalit�t}}
		\tabularnewline
		\tabucline[2pt]{-}
		\multicolumn{3}{l}{\cellcolor{bluegray}  Qualit�tsmerkmal:  Funktionalit�t }
		\tabularnewline
		\cellcolor{testblau}  NREQ0.01 - Sicherheit &
		\cellcolor{testblau}  Conditional &
		\cellcolor{testblau}High
		\tabularnewline
		\tabucline[2pt]{-}
		\multicolumn{3}{l}{\cellcolor{bluegray}  Qualit�tsmerkmal:  Zuverl�ssigkeit  }
		\tabularnewline
		\cellcolor{testblau}  NREQ1.01 - Fehlertoleranz  &
		\cellcolor{testblau}Conditional &
		\cellcolor{testblau}High
		\tabularnewline		
		\cellcolor{testblau}  NREQ1.02 - Wiederherstellbarkeit  &
		\cellcolor{testblau}Essential &
		\cellcolor{testblau}High
		\tabularnewline	
		\tabucline[2pt]{-}
		\multicolumn{3}{l}{\cellcolor{bluegray}  Qualit�tsmerkmal:  Effizienz  }
		\tabularnewline
		\cellcolor{testblau}  NREQ3.01 - Effizient in Programmierung  &
		\cellcolor{testblau} Essential &
		\cellcolor{testblau} High
		\tabularnewline		
		\cellcolor{testblau}  NREQ3.02 - Effizient in Installation  &
		\cellcolor{testblau} Essential &
		\cellcolor{testblau}High
		\tabularnewline
		\tabucline[2pt]{-}
		\multicolumn{3}{l}{\cellcolor{bluegray}  Qualit�tsmerkmal:  Wartbarkeit  }
		\tabularnewline
		\cellcolor{testblau}  NREQ4.01 - Einfach erweiterbar/�nderbar &
		\cellcolor{testblau} Essential &
		\cellcolor{testblau} High
		\tabularnewline		
		\cellcolor{testblau}  NREQ4.02 - Stabilit�t  &
		\cellcolor{testblau} Optional &
		\cellcolor{testblau} High
		\tabularnewline
		\cellcolor{testblau}  NREQ4.03 - Testbarkeit  &
		\cellcolor{testblau} Essential &
		\cellcolor{testblau} High
		\tabularnewline
		\cellcolor{testblau}  NREQ4.04 - Analysierbarkeit    &
		\cellcolor{testblau} Optional &
		\cellcolor{testblau} High
		\tabularnewline		
		
		
	\end{longtabu}\end{center}

\subsubsection{Spring}
Spring ist ein MVC Framework in Java. Man programmiert in der J2EE Umgebung und bietet eine API zum Client. Gleichzeitig sendet man den AngularJS Stack zum Client, welcher anschliessend die API konsumiert. Als Persistance Layer k�nnen Relationale Datenbanken wie MySQL, Oracle oder Sybase verwendet werden. �ber Data Access Object wird dieser Layer angesprochen und in Models Emuliert. 

\paragraph{Installation und Konfiguration}
Die Installation und Konfiguration von Spring mit Ember.js ist komplex. Die Installation funktioniert �ber Maven, einzelne Komponenten nach der Installation m�ssen aufeinander konfiguriert werden. Security sowie ORM f�r den Persistance Layer sind nicht im Spring Package enthalten und m�ssen zus�tzlich hinzugef�gt und auf Spring konfiguriert werden. Es gibt Templates, welche dies etwas einfacher gestalten, jedoch in meinen Recherchen habe ich kein Funktionierendes Modell gefunden

\paragraph{Portabilit�t} Durch Maven ist eine Spring installation einfach auf einer anderen Maschiene installierbar. Einmal eingerichtet ist es einfach m�glich den Code auszutauschen und die Applikation laufen zu lassen.

\paragraph{Funktonalit�t} Out of the Box bringt Spring keine Funktionalit�t ausser den MVC Workflow. Es gibt kein Basis User Management, kein Front End MVC, keine CSS Frameworks oder sonstiges. Ember.js muss zus�tzlich selber installiert und integriert werden.
\textcolor{red}{QUELLEN!!???}
\paragraph{Stabilit�t} Spring gilt das eines der meist eingesetzten Frameworks auf dem Markt. Es ist bekannt f�r seine Stabilit�t und Enterprise Readiness.  

\paragraph{Skalierbarkeit} Spring ist skalierbar innerhalb des Application Servers. Es gibt jedoch limitationen im Clustering.  

\paragraph{Erweiterbarkeit} Die Programmiersprache von Spring ist Java oder Scala. Diese Programmiersprachen bieten eine breite Palette an Funktionen und Objekten, welche benutzt werden k�nnnen. Spring ist somit sehr gut erweiterbar.

\paragraph{Stacks} Ein Stack f�r Spring, der alle Plug-Ins und Client MVCs mitbringt heisst JHipster \footcite{jhipster}. Dieser Stack bringt eine Userverwaltung, Integration von Persistance Layer, Client MVC (AngularJS) und vieles weiter. Die Installation ist komplex und schwer verst�ndlich. 

\paragraph{Objekte / API} Ein integraler Bestandteil der Applikation soll eine API mit JSON sein. Bei Spring muss man Javaobjekte zu JSON objekte serialisieren und umgekehrt. Folgender Workflow existiert f�r Objekte von Persistance Layer bis zum Ausgang der API:
\begin{enumerate}
	\itemsep -0.4em
	\item Relationale Datenbankfelder
	\item Konvertierung zu einem Objekt �ber ein ORM(Object-relational Mapping) Interface
	\item Konvertierung von Objekt zu einem JSON-Objekt
	\item JSON Objekt wird versendet
\end{enumerate}

Diese Konversionen m�ssen oft manuell erstellt werden und sind negativ um effizient zu programmieren.

\newpage
\subsubsection{NodeJS / ExpressJS}

\paragraph{Installation und Konfiguration} ExpressJS wird �ber npm installiert. Die Installation ist sehr simpel. Zus�tzlich existieren f�r NodeJS und ExpressJS eine grosse Anzahl Stacks, welche sehr einfach zu installieren sind. Der Persistance Layer benutzt Mongodb, d.h. Objekte werden nativ im JSON Format in der Datenbank gespeichert. Konfigurationen �ber die Stacks sind sehr intuitiv.

\paragraph{Portabilit�t} NPM l�sst sich wie Maven so konfigurieren, das eine automatische Installation ausgef�hrt wird der ben�tigten Plug-ins. Die Portabilit�t ist somit wie bei Spring sehr gut.

\paragraph{Funktionalit�t} NodeJS bzw. ExpressJS besitzten out of the box keine Funktionalit�t. Diverse Stacks unterst�zten jedoch Userverwaltung, Plugin-Handling, Client sowie Server MVCs.

\paragraph{Stabilit�t} NodeJS ist nicht so viel benutzt wie Spring. Die Stabilit�t ist somit nicht endg�ltig bewiesen. einige grosse Firmen setzten jedoch schon NodeJS ein und es ist bis jetzt noch nichts bekannt �ber Probleme mit der Stabilit�t.

\paragraph{Skalierbarkeit} NodeJS ist sehr gut skalierbar. �ber Loadbalancer kann man HTTP Anfragen an mehrere NodeJS prozesse verteilen. Da der ganze Kontext im Persistance Layer existiert  - und MongoDB im Cluster l�uft - ist NodeJS super skalierbar.

\paragraph{Erweiterbarkeit} JavaScript ist sehr popul�r und es existieren Zahlreiche Bibliotheken. Stacks sind in Module aufgebaut und die Erweiterbarkeit und Wartbarkeit von Code ist sehr effizient. 

\paragraph{Stacks}
In NodeJS ist der MEAN Stack weit verbreitet. Der MEAM Stack besteht aus folgenden Produkten:
\begin{itemize}
		 \itemsep-0.5em
	\item M - MongoDB, der Skalierbare Persistance Layer
	\item E - ExpressJS, ein MVC um APIs zu entwickeln
	\item A - AngularJS, ein MVC auf dem Client um Sing-Page Applikationen zu erstellen, welche auf die ExpressJS API zugreifen.
	\item N - NodeJS, JavaScript Applikationsserver, welcher sehr gut skalierbar ist.
\end{itemize}

\paragraph{Objekte / API} Ein integraler Bestandteil der Applikation soll eine API mit JSON sein. Bei einem Mean Stack gibt es keine Konversionen. Von der Datenbank bis zum Output der API existiert immer genau das gleiche Objekt.

\subsubsection{Evaluation}
F�r die Stacks MEAN sowie JHipster wird nun eine Evaluation durchgef�hrt. Es soll herausgefunden werden, welcher Stack besser geeignet ist f�r eine effiziente Programmierung.

Die nicht funktionalen Anforderungen werden somit mit einer Bewertung von 1-10 f�r jedes Framework versehen. Die Bewertung entsteht auf Basis der Erfahrung des Autors dieser Arbeit aufgrund von Recherchen. Argumente sind in dem Kapitel der einzelnen Frameworks aufgelistet.
\begin{center}
	\tabulinesep = 1mm
	\begin{longtabu} to \linewidth [m]{X[1, m , l]|[2pt]p{2.95cm}|[2pt]p{2.95cm}|[2pt]}
		\arrayrulecolor{white}
		
		\tabucline[2pt]{-}
		\textcolor{white}{\textbf{\cellcolor{airforceblue}Anforderung}}  &  
		\textcolor{white}{\textbf{\cellcolor{airforceblue}MEAN}}&
		\textcolor{white}{\textbf{\cellcolor{airforceblue}JHipster}}
		\tabularnewline
		\tabucline[2pt]{-}
		\multicolumn{3}{l}{\cellcolor{bluegray}  Qualit�tsmerkmal:  Funktionalit�t }
		\tabularnewline
		\cellcolor{testblau}  NREQ0.01 - Sicherheit &
		\cellcolor{testblau}  9 &
		\cellcolor{testblau}10
		\tabularnewline
		\tabucline[2pt]{-}
		\multicolumn{3}{l}{\cellcolor{bluegray}  Qualit�tsmerkmal:  Zuverl�ssigkeit  }
		\tabularnewline
		\cellcolor{testblau}  NREQ1.01 - Fehlertoleranz  &
		\cellcolor{testblau}10 &
		\cellcolor{testblau}9
		\tabularnewline		
		\cellcolor{testblau}  NREQ1.02 - Wiederherstellbarkeit  &
		\cellcolor{testblau}10 &
		\cellcolor{testblau}9
		\tabularnewline	
		\tabucline[2pt]{-}
		\multicolumn{3}{l}{\cellcolor{bluegray}  Qualit�tsmerkmal:  Effizienz  }
		\tabularnewline
		\cellcolor{testblau}  NREQ3.01 - Effizient in Programmierung  &
		\cellcolor{testblau} 10 &
		\cellcolor{testblau} 6
		\tabularnewline		
		\cellcolor{testblau}  NREQ3.02 - Effizient in Installation  &
		\cellcolor{testblau} 10 &
		\cellcolor{testblau}7
		\tabularnewline
		\tabucline[2pt]{-}
		\multicolumn{3}{l}{\cellcolor{bluegray}  Qualit�tsmerkmal:  Wartbarkeit  }
		\tabularnewline
		\cellcolor{testblau}  NREQ4.01 - Einfach erweiterbar/�nderbar &
		\cellcolor{testblau} 10 &
		\cellcolor{testblau} 7
		\tabularnewline		
		\cellcolor{testblau}  NREQ4.02 - Stabilit�t  &
		\cellcolor{testblau} 8&
		\cellcolor{testblau} 10
		\tabularnewline
		\cellcolor{testblau}  NREQ4.03 - Testbarkeit  &
		\cellcolor{testblau} 10 &
		\cellcolor{testblau}10 
		\tabularnewline
		\cellcolor{testblau}  NREQ4.04 - Analysierbarkeit    &
		\cellcolor{testblau} 9 &
		\cellcolor{testblau} 9
		\tabularnewline		
		\tabucline[2pt]{-}
		\cellcolor{bluegray}  Total &
		\cellcolor{bluegray} 86 &
		\cellcolor{bluegray} 77
	\end{longtabu}\end{center}
	
Die Applikation wird somit mit dem MEAN Stack entwickelt. 

\subsection{Kontext}
Die Applikation ist aufgeteilt auf einen Server, sowie auf einen Client, welcher ein Browser oder eine Android App ist. Auf dem Server sind alle Daten hinterlegt:
\begin{itemize}
	\itemsep -0.5em
	\item Gespeicherte Objekte in MongoDB
	\item Server Logik
	\item Client Daten, welche vom Browser �ber HTTP abgefragt werden
\end{itemize}

Im Anfangszustand hat der Client keine Daten. Der Client bekommt die Daten bei dem Abruf der Applikations URL �ber HTTP. Er baut nun die Logik im Browser Cache auf und startet das JavaScript Programm. Das JavaScript Programm l�dt nun die auf dem Server gespeicherten Objekte �ber HTTP AJAX Abrufe und stellt diese dar.  

Die Logik von Server wie auch Client benutzt das M(V)C Pattern. Objecte werden in Models - inklusive Business Logik -  gespeichert, der Controller beinhaltet die Applikationslogik, welche das Model sowie die View ausw�hlt. Die View rendert nun das Model in ein bestimmtes Schema (siehe Abbildung \ref{DetAppArch}). 
\begin{figure}[ht]
	\centering
	\includegraphics[width=0.9\textwidth]{Graphics/DetailAppArch.png}
	\caption{Detaillierte Applikations Architektur}
	\label{DetAppArch}
\end{figure}

\subsubsection{Continues Integration}
Die Applikation wird auf dem Server des Autors gehostet. 

Um Updates der Applikation direkt zu installieren besitzt der Server eine Applikation f�r Continues Integration namens Jenkins. Jenkins f�hrt ein Deployment der Applikation automatisch aus bei Anforderung eines solchen Deployments. In Jenkins werden diese Deployments Jobs genannt. 

Dieser Jenkins Job wird bei jedem Push nach Github ausgef�hrt. Das heisst bei jedem Update der Applikation wird die neuste Version direkt auf dem Server unter der Adresse https://racket.marques.pw aktualisiert. 




\subsection{Businessschicht}
\subsubsection{DB Design}
In der Applikation gibt es kein Relationales Datenbankmodel. MongoDB arbeiet mit Dokumenten, sowie Referenzen. Dokumente sind JSON-Objekte in JavasScript, welche in MongoDB als Dokument gespeichert werden. Ein Objekt ist eine Representation von Business Objekten in der Applikation.

Das User-Objekt repr�sentiert der User der Applikation. Der User hat einen Namen, einen Usernamen, ein Passwort (encrypted und salted), Berechtigungen und Freunde. Zus�tzlich werden Ihm andere Objekte zugeordnet sowie andere User (Repr�sentation als Freund).

Das Court-Objekt repr�sentiert ein Racketsportzentrum der Applikation. Dieses Objekt wird ben�tigt um den Physikalischen Austragungsort eines Spieles zu definieren. Das Objekt beinhaltet einen Namen, eine Adresse (inklusive Koordinaten f�r eine zuk�nftige Umkreissuche), was f�r Sportarten gespielt werden k�nnen und welche User in diesem Racketsportzentrum spielen wollen.

Das Match-Objekt resp�sentiert das Spiel, welches geplant, ausgetragen oder beendet ist. Das Spiel-Modell definiert zwei oder einen Spieler, einen Status, eine Sportart, ein Court, mehrere Datumvorschl�ge, maximal fixes Datum, eine Punkzahl, sowie ein Gewinner. Hinter dem Match-Objekt existiert ein relativ grosser Business-Workflow, welcher im Kapitel <<<<IMPLEMENTATION Matchmaking>>>> definiert ist. 

Das Liga-Objekt repr�sntiert eine Liga. Verschiedene Benutzer k�nnen einer Liga beitereten und sind nach Beitritt bestimmten regeln unterworfen. Daf�r k�nnen die Benutzer spiele f�r die Liga spielen und so Punkte f�r einen optionalen Preis sammeln. Die Liga beinhaltet neben einem Namen, einer Sportart, einem Standort (inklusive Koordinaten, f�r zuk�nftige Umkreissuche), einem Niveau, Start- und Enddatum, einem Preis und einem Matchmaking Plan (wird sp�ter im Dokument erl�uert>>>>>>>REF). 

Folgende Grafik zeigt die Beziehung der Verschiedenen Schemas auf, das Datenbankmodell ist nicht Relational, und somit nicht normalisiert. 
\begin{figure}[ht]
	\centering
	\includegraphics[width=0.9\textwidth]{Graphics/DBModel.png}
	\caption{Datenbank Modell}
	\label{DBMOdel}
\end{figure}

\newpage
\subsection{Pr�sentationsschicht} 
Als GUI wird ein standard Bootstrap Design verwendet. Ohne Authentisierung kann nur die Home-Page gesehen werden sowie die Login und Signup Page. F�r alle anderen Seiten muss der Benutzer authentisiert sein. Sobald die Authentisierung durchgef�hrt wurde, gibt es k�nnen die Elemente (Nach Datenbank) Benutzer, Liga, Racketsportzentrum sowie Spiele selektiert werden. Innerhalb der einzelnen Menu kann man verschiedene Operationen direkt ansteuern, einige nur �ber andere Operationen. Folgendes Diagram zeigt die Interaktion durch die verschiedenen Views.
\begin{figure}[ht]
	\centering
	\includegraphics[width=0.9\textwidth]{Graphics/GUIInteraction.png}
	\caption{GUI Interaktions Modell}
	\label{GUIInteraction}
\end{figure}

\subsubsection{User Section}
Die User Section einhaltet drei Views die direkt aus dem Menu erreichbar sind. Die erste View User Profile erm�glicht dem User, Details �ber sich preiszugeben. Er kann zus�tzlich das Passwort �ndern. Inder Friends view kann er neue Friendrequests erstellen, und pendente Friendrequests annehmen oder ablehnen. Die Social Account View bietet eine Verkn�pfung von Social Accounts mit der Applikation an.

\subsubsection{Court Section}
Die Court Section beinhaltet vier Views sowie eine Aktion. In der new Court View kann ein neues Racketsportzentrum registriert werden. In der List Courts view findet man alle  Racketsportzentren und erreicht bei klick auf ein Zentrum die View Court Details View. In dieser View kann man alle Details des Racketsportzentrum anschauen, sowie alle Spieler, welche in diesem Racketsportzentrum spielen. Durch klick auf den Spieler kann in die New Match View gewechselt werden, um einen Spieler herauszufordern. Von der Detail View kann man zus�tzlich das Court l�schen, sofern man das Court erstellt hat oder ein Admin ist.

\subsubsection{League Section}
Die League Section beinhaltet vier Views sowie eine Aktion. In der New League View kann ein neues Liga registriert werden. In der List League View findet man alle  Ligen und erreicht bei klick auf eine Liga die View League Details View. In dieser View kann man alle Details die Liga anschauen, sowie alle Spieler, welche in dieser Liga spielen. Durch klick auf den Spieler kann in die New Match View gewechselt werden, um einen Spieler herauszufordern. Von der Detail View kann man zus�tzlich die Liga l�schen, sofern man die Liga erstellt hat oder ein Admin ist.

\subsubsection{Match Section}
Die Match Section beinhaltet alle Interaktionen im Match. Drei Views sind direkt aus dem Menu erreichbar. Auf der New Match View kann man ein neues Spiel erstellen. Man kann Court, Spieler in einem Formular ausw�hlen. �ber den Menupunkt New Broadcast Match View, w�hlt man ein Court sowie eine Zeit und alle Spieler, welche in diesem Court spielen werden angefragt f�r eine spontanes Spiel. In der List Match View werden alle Spiele aufgelistet. Von da kommt man in die View Match Details View, welche den Matchworkflow abdeckt.

\subsubsection{GUI Design}
Als Design wurde Bootstrap benutzt. Das Design besteht aus einem Header sowie einem Hauptfenster. 

\begin{figure}[ht]
	\centering
	\includegraphics[width=0.9\textwidth]{Graphics/website.png}
	\caption{GUI Design}
	\label{GUIInteraction}
\end{figure}
% !TeX spellcheck = de_CH
\chapter{Implementation}
\section{REST API}
Die REST API besteht aus f�nf Endpunkten:
\begin{itemize}
	\itemsep-0.8em
	\item /matches - Stellt alle Operationen f�r Matches zur Verf�gung
	\item /leagues - Stellt alle Operationen f�r Liga Management zur Verf�gung
	\item /courts - Stellt alle Operationen f�r die Verwaltung von Racketsportzentren zur Verf�gung
	\item /users - Stellt alle Operationen f�r das Usermanagemnt zur Verf�gung
	\item /core - Stellt Core-Funktionalit�ten (Home Seite) zur Verf�gung
\end{itemize}
Die Endpunkte /users und /core waren im MEANJS Stack schon vorhanden. Der User Endpunkt wurde jedoch modifiziert. Die Modifizierungen sind in dem Kapitel dokumentiert, die schon vorhandenen Endpunkte nicht. 





\subsection{Spiel Endpunkt /matches}
\begin{figure}[ht]
	\centering
	\includegraphics[width=0.8\textwidth]{Graphics/api_match.png}
	\caption{API f�r Spiele}
	\label{MatchAPI}
\end{figure}

Bei allen Match Endpunkten muss der User als Spieler registriert sein, um Informationen �ber das Spiel zu erhalten. Ausnahme ist, wenn er direkt die ID eingibt und direkt auf Spiel Detauls zugreift. 

Mit einem Post f�gt man der Datenbank ein Spiel hinzu, mit PUT aktualisiert man das Spiel mit neuen oder ge�nderten Daten. Hinter dem PUT interface gibt es gewisse Input Validations um Missbrauch zu verhindern. In dieser ersten Version sin ddie Validations jedoch relativ einfach gehalten. 

Um eine gute �bersicht aller Spiele auf der List Matches View zu erstelllen, gibt es f�r jeden Status eines Spieles einen eigenen API Call (/matches/new, /matches/open, /matches/inprogress, /matches/proposed, /matches/r2c, /matches/done). 

Der API Endpunkt /matches/broadcast listet zus�tzlich alle broadcasting Anfragen auf. Der Conntroller des Enpunkt Korreliert, in welchen Racketsportzentren der User registriert ist und die Matches ohne zweiten Spieler und gibt das Resultat dem Client.

\subsection{Liga Endpunkt /leagues}

\begin{figure}[ht]
	\centering
	\includegraphics[width=0.8\textwidth]{Graphics/api_league.png}
	\caption{API f�r Ligen}
	\label{LeagueAPI}
	\end{figure}

Bei der Liga gibt es - wie bei allen Endpunkten - CRUD Endpunkte (/leagues f�r list all, /leagues/:id f�r Show Element, POST /leagues f�r Create League, PUT /leagues/:id f�r Update League). Zus�tzlich gbit es eine Join Action, welche den authentisierten User einer Liga hinzuf�gt sowie ein Leave Endpunkt um die Registriertung zu l�schen.

Ein zus�tzlicher Endpunkt ist /leagues/invite, welcher erm�glicht einen User zu einer Liga einzuladen.
	\newpage
\subsection{Racketsportzentrum Endpunkt /courts}
\begin{figure}[ht]
	\centering
	\includegraphics[width=0.8\textwidth]{Graphics/api_court.png}
	\caption{API f�r Racketsportzentren}
	\label{CourtAPI}
\end{figure}
Gleich wie beim Liga Endpunkt gibt es die CRUD Endpunkte sowie ein Join/Leave Endpunkt

\subsection{Benutzer Endpunkt /users}
Neben den �blichen Benutzerverwaltungs Endpunkten (/user/signin, /user/signout, /user signup, /auth/forgot), welche hier nicht Dokumentiert werden, gibt es Endpunkte f�r das Freunde-System:
\begin{itemize}
	\itemsep-0.5em
	\item GET /users/friend - Auflistung aller Freunde
	\item DELETE /users/friend - L�schen eines Freundes
	\item GET /users/request -  Senden eines Freund Requests
	\item DELETE /users/request - L�schen eines Freund Requests
\end{itemize}

\section{Web Applikation}


\section{Android Applikation}
Als Grundlage f�r die Android Applikation wurde eine Applikation von gonative.io generiert. Im Laufe des Projektes - nach erheblicher �berschreitung des vorgeschriebenen Aufwandes - wurde entschieden keine vollst�ndig Native Webapplikation zu erstellen. Stattdessen wird eine WebView erstellt, welche die Mobile Webseite darstellt. Um alle Funktionalit�t zu behalten, wird �ber die WebView und Interception Algorithmen Push-Nachriten erm�glicht. Der einzige Setback ist, das die Website offline nicht verf�gbar ist.



\section{Workflows}
\section{Allgemeine Workflows}
\subsection{CRUD f�r Datenobjekte}
Alle Datenobjekte haben einen Endpunkt. Jeder Endpunkt stellt CRUD Operationen zur Verf�gung:
\begin{itemize}
	\itemsep -0.5em
	\item C - Neues Objekt erstellen
	\item R - Ein Objekt anzeigen
	\item U - Ein Objekt aktualisieren
	\item D - Ein Objekt l�schen
	\end{itemize}
	
	Zus�tzlich wird noch einen Endpunkt zur Auflistung aller Objekte angeboten. 

\newpage
\section{Court}
\subsection{Court Registrierung}
Wenn der User den Knopf im User Interface zur Registrierung das Racketsportzentrums dr�ckt, wird im Hintergrund der /courts/join API Call ausgef�hrt. Dieser Call f�gt der User der Anfrage in ein Array - bestehend aus allen registrierten Usern - ein. 
\begin{figure}[ht]
	\centering
	\includegraphics[width=0.7\textwidth]{Graphics/workflow_court.png}
	\caption{Racketsportzentrum Workflow}
	\label{CourtWorkflow}
\end{figure}
\newpage

\section{Liga}
\subsection{Liga Registrierung}
Identisch zu der Court Registrierung funktioniert die Liga Registrierung
\begin{figure}[ht]
	\centering
	\includegraphics[width=0.7\textwidth]{Graphics/workflow_league.png}
	\caption{Liga Workflow}
	\label{LegueWorkflow}
\end{figure}
	\newpage
	\subsection{Automatische Herausforderung Liga}
	
	Bei der erstellung einer Liga kann ausgew�hlt werden ob automatische Herausforderungen aktiviert werden sollten. Aktuell gibt es vier verschiedene ausw�hlbare Modi:
	\begin{itemize}
		\item Weeklyall: W�chentliche Herausforderung, jeder gegen jeder, zuf�lliger Gegner
		\item Biweeklyall: Herausforderung alle zwei Wochen, jeder gegen  jeder, zuf�lliger Gegner
		\item Weekly 
	\end{itemize}
	\begin{figure}[ht]
		\centering
		\includegraphics[width=0.7\textwidth]{Graphics/schedue_workflow.png}
		\caption{Schedule Workflow}
		\label{ScheduleWorkflow}
	\end{figure}
\begin{landscape}

\section{Match}
\subsection{Match Workflow}
Der Matchworkflow ist das Hauptelement der Applikation. Der Workflow regelt, wie der Match als Business Prozess durchgef�hrt wird. 
\begin{figure}[ht]
	\centering
	\includegraphics[width=1.3\textwidth]{Graphics/match_workflow.png}
	\caption{Spiel Workflow}
	\label{MatchWorkflow}
\end{figure}
\end{landscape}

Der Workflow wird �ber drei verschiedene F�lle gestartet: 
\begin{itemize}
	\itemsep -0.5em
	\item Das System erstellt Auto-Herausforderungen f�r die Liga
	\item Der User erstellt ein Broadcast Spiel
	\item Der User erstellt ein regul�res spiel.
	
	\end{itemize}
	
Wenn der User ein \textbf{regul�res Spiel} erstellt, sind beide Spieler, sowie Terminvorschl�ge schon definiert. Es folgt die Aktion \grqq Termin ausw�hlen\grqq.

Erstellt der User ein \textbf{broadcast Spiel}, hat das Spiel den Status \grqq New\grqq, jedoch noch keinen zweiten Spieler definiert. Zus�tzlich werden keine Terminvorschl�ge ausgef�llt, sondern einen fixen Termin. Akzeptiert jemand den Broadcast wird der zweite Spieler eingetragen und der Status �ndert sich direkt auf Open.
 
Sind User in einer Liga, erstellt die \textbf{Liga eine Herausforderung}. Das Spiel enth�lt kein Court und keine Terminvorschl�ge. Der User muss nun Terminvorschl�ge ausf�llen und ein Court definieren. 

Anschliessend haben alle Use Cases den gleichen Workflow. Ist das Spiel und der Termin definiert. Geht der Status des Spiels zu \grqq Open\grqq . Danach kann von beiden Spielern der Status auf \grqq In progress\grqq gesetzt werden. Beide k�nnen ein Resultat eintragen. Der jeweil andere Spieler best�tigt anschliessend das Resultat. Bei der Best�tigung des Resultats wird das Spiel archiviert und optional die Rangliste der Liga aktualisiert.


\section{User}
\subsection{Freunde System}
% !TeX spellcheck = de_CH
\chapter{Test}
\section{Unit Tests}
Der MEAN Stack bietet ein Framework direkt an. Es wird unterschieden zwischen dem Client Test Framework und dem Server Test Framework.

Als Client Test Framework wird Karma --- ein Framework, um AngularJS zu testen --- verwendet. F�r jeden Controller werden die verschiedenen Methoden getestet.  Als Beispiel testet dieser Test, ob die find()-Methode gew�nscht funktioniert:
\begin{lstlisting}
it('$scope.find() should create an array with at least one Court object fetched from XHR', inject(function(Courts) {
	// Create sample Court using the Courts service
	var sampleCourt = new Courts({
		'name':'Vitis','address':'Vitis','contact':'000000000','sports':['Squash','Tennis','Badminton','Tabletennis']
	});

	// Create a sample Courts array that includes the new Court
	var sampleCourts = [sampleCourt];

	// Set GET response
	$httpBackend.expectGET('courts').respond(sampleCourts);

	// Run controller functionality
	scope.find();
	$httpBackend.flush();

	// Test scope value
	expect(scope.courts).toEqualData(sampleCourts);
}));
\end{lstlisting}

Zuerst wird ein Beispiel-Court-Objekt erstellt (Zeile 3--5). Auf Zeile 11 wird ein Mock-Objekt erstellt, das den API Call /courts abf�ngt, und das Beispiel-Court zur�cksendet. Anschliessend wird (Zeile 14) die Methode aufgerufen und auf Zeile 18 die Antwort mit dem Anfang verglichen. Sind die Daten gleich, ist der Test f�r die Methode gegl�ckt.

Als Server Test Framework wird Mocha --- eine NodeJS/ExpressJS Testsuite verwendet. Models sowie Controller werden hier getestet. 

Als Beispiel testet dieser Code, ob die Datenfelder richtig spezifiziert sind, und verschiedene Save-Funktionen funktionieren oder einen Fehler zur�ckgeben:
\begin{lstlisting}
describe('Method Save', function() {
	it('should be able to save without problems', function(done) {
		return court.save(function(err) {
			should.not.exist(err);
			done();
		});
	});

	it('should be able to show an error when try to save without name', function(done) { 
		court.name = '';

		return court.save(function(err) {
			should.exist(err);
			done();
		});
	});
	it('should be able to show an error when try to save without coordinates', function(done) {
		court.lat = '';

		return court.save(function(err) {
			should.exist(err);
			done();
		});
	});
});
\end{lstlisting}
Der erste Test pr�ft, ob ein valides (zuvor erstelltes) Court-Objekt  problemlos gespeichert werden kann. Daf�r wird die court.save() Funktion aufgerufen und gepr�ft, ob es einen Error gibt.

Anschliessend (ab Zeile 9) wird der Name oder die Koordinaten (ab Zeile 17) gel�scht, was nicht zul�ssig ist. Das Resultat bei dem Speichern des Objekt sollte ein Error sein. Ist dem so, ist der Test erfolgreich.

Bei den Controller Tests wird die Applikation in der Test-Umgebung gestartet. Verschiedene Objekte werden bei Teststart in die Test-Datenbank eingef�gt:
\begin{lstlisting}
....
beforeEach(function(done) {
		// Create user credentials
		credentials = {
			username: 'username',
			password: 'password'
		};

		// Create a new user
		user = new User({
			firstName: 'Full',
			lastName: 'Name',
			displayName: 'Full Name',
			email: 'test@test.com',
			username: credentials.username,
			password: credentials.password,
			provider: 'local'
		});

		// Save a user to the test db and create new Court
		user.save(function() {
....
\end{lstlisting}
Anschliessend wird die Applikation mit verschieden Requests angesprochen und die Response auf Validit�t �berpr�ft:
\begin{lstlisting}
it('should not be able to save Court instance if not logged in', function(done) {
		agent.post('/courts')
			.send(court)
			.expect(401)
			.end(function(courtSaveErr, courtSaveRes) {
				// Call the assertion callback
				done(courtSaveErr);
			});
	});
\end{lstlisting}
In diesem Beispiel wird versucht ein Court anzulegen, ohne vorher einzuloggen. Ein Access Denied Fehler muss von der API zur�ckgesendet werden (siehe Zeile 4).

\section{System Tests}
Zus�tzlich zu den Unit Tests werden die Anforderungen manuell getestet mithilfe eines Systemtests. Zuerst werden Testf�lle f�r Anforderungen erstellt, diese ausgef�hrt und das erwartete Ergebnis mit dem tats�chlichen verglichen. 
\newpage
\subsection{Testf�lle}
\begin{center}
	\tabulinesep = 1mm
	\begin{longtabu} to \linewidth [m]{|[2pt]p{3cm}|[2pt]X[1, m , l]|[2pt]}
		\arrayrulecolor{white}	
		\tabucline[2pt]{-}
		\textcolor{white}{\textbf{\cellcolor{airforceblue}TE0.01}}  &   
		\textcolor{white}{\textbf{\cellcolor{airforceblue}Erfolgreiche Anmeldung an das System }}
		\tabularnewline
		\tabucline[2pt]{-}
		\textcolor{white}{\textbf{\cellcolor{airforceblue}Beschreibung}} &  
		\cellcolor{testblau}{Dieser Test pr�ft, ob eine Anmeldung an das System funktioniert mit g�ltigen Anmeldedaten. }
		\tabularnewline
		\tabucline[2pt]{-}
		\textcolor{white}{\textbf{\cellcolor{airforceblue}Version}} &
		\cellcolor{testblau} 1.0
		\tabularnewline
		\tabucline[2pt]{-}
		\textcolor{white}{\textbf{\cellcolor{airforceblue}Getestete Anforderung}} &
		\cellcolor{testblau} REQ0.01 --- Anmeldung an das System
		\tabularnewline
		\tabucline[2pt]{-}
		\textcolor{white}{\textbf{\cellcolor{airforceblue}Testschritte }} &
		\cellcolor{testblau}
		\begin{enumerate}
		\itemsep -0.4em
			\item Navigieren zur Anmeldeseite (webserver/signin)
			\item Eingeben von g�ltigen Anmeldedaten
		\end{enumerate}
		\tabularnewline
		\tabucline[2pt]{-}
		\textcolor{white}{\textbf{\cellcolor{airforceblue}Erwartetes Resultat}} &
		\cellcolor{testblau}	
		Erfolgreicher Login, Weiterleitung zur Startseite. 
	\end{longtabu}
\end{center}
\begin{center}
	\tabulinesep = 1mm
	\begin{longtabu} to \linewidth [m]{|[2pt]p{3cm}|[2pt]X[1, m , l]|[2pt]}
		\arrayrulecolor{white}	
		\tabucline[2pt]{-}
		\textcolor{white}{\textbf{\cellcolor{airforceblue}TE0.02}}  &   
		\textcolor{white}{\textbf{\cellcolor{airforceblue}Fehlgeschlagene Anmeldung an das System }}
		\tabularnewline
		\tabucline[2pt]{-}
		\textcolor{white}{\textbf{\cellcolor{airforceblue}Beschreibung}} &  
		\cellcolor{testblau}{Dieser Test pr�ft, ob eine Anmeldung an das System fehlschl�gt mit falschen Anmeldedaten }
		\tabularnewline
		\tabucline[2pt]{-}
		\textcolor{white}{\textbf{\cellcolor{airforceblue}Version}} &
		\cellcolor{testblau} 1.0
		\tabularnewline
		\tabucline[2pt]{-}
		\textcolor{white}{\textbf{\cellcolor{airforceblue}Getestete Anforderung}} &
		\cellcolor{testblau} REQ0.01 --- Anmeldung an das System
		\tabularnewline
		\tabucline[2pt]{-}
		\textcolor{white}{\textbf{\cellcolor{airforceblue}Testschritte }} &
		\cellcolor{testblau}
		\begin{enumerate}
		\itemsep -0.4em
			\item Navigieren zur Anmeldeseite (webserver/signin)
			\item Eingeben von ung�ltigen Anmeldedaten
		\end{enumerate}
		\tabularnewline
		\tabucline[2pt]{-}
		\textcolor{white}{\textbf{\cellcolor{airforceblue}Erwartetes Resultat}} &
		\cellcolor{testblau}	
		Fehlgeschlagener Login, Fehlermeldung (z.B. Wrong Username/Password)
	\end{longtabu}
\end{center}
\begin{center}
	\tabulinesep = 1mm
	\begin{longtabu} to \linewidth [m]{|[2pt]p{3cm}|[2pt]X[1, m , l]|[2pt]}
		\arrayrulecolor{white}	
		\tabucline[2pt]{-}
		\textcolor{white}{\textbf{\cellcolor{airforceblue}TE0.03}}  &   
		\textcolor{white}{\textbf{\cellcolor{airforceblue}Erfolgreiche Anmeldung an das System mittels OAuth}}
		\tabularnewline
		\tabucline[2pt]{-}
		\textcolor{white}{\textbf{\cellcolor{airforceblue}Beschreibung}} &  
		\cellcolor{testblau}{Dieser Test pr�ft, ob eine Anmeldung an das System mit OAuth funktioniert. }
		\tabularnewline
		\tabucline[2pt]{-}
		\textcolor{white}{\textbf{\cellcolor{airforceblue}Version}} &
		\cellcolor{testblau} 1.0
		\tabularnewline
		\tabucline[2pt]{-}
		\textcolor{white}{\textbf{\cellcolor{airforceblue}Getestete Anforderung}} &
		\cellcolor{testblau} REQ0.03 --- Authentisierung �ber OAuth
		\tabularnewline
		\tabucline[2pt]{-}
		\textcolor{white}{\textbf{\cellcolor{airforceblue}Testschritte }} &
		\cellcolor{testblau}
		\begin{enumerate}
		\itemsep -0.4em
			\item Navigieren zur Anmeldeseite (webserver/signin)
			\item Auf das OAuth Symbol von Google klicken
			\item Bei Google anmelden
			\item Autorisierung (wenn noch nicht gemacht) freigeben
		\end{enumerate}
		\tabularnewline
		\tabucline[2pt]{-}
		\textcolor{white}{\textbf{\cellcolor{airforceblue}Erwartetes Resultat}} &
		\cellcolor{testblau}	
		Erfolgreicher Login, Weiterleitung zur Startseite.
	\end{longtabu}
\end{center}
\begin{center}
	\tabulinesep = 1mm
	\begin{longtabu} to \linewidth [m]{|[2pt]p{3cm}|[2pt]X[1, m , l]|[2pt]}
		\arrayrulecolor{white}	
		\tabucline[2pt]{-}
		\textcolor{white}{\textbf{\cellcolor{airforceblue}TE0.04}}  &   
		\textcolor{white}{\textbf{\cellcolor{airforceblue}Erfolgreiche Registrierung mit dem System}}
		\tabularnewline
		\tabucline[2pt]{-}
		\textcolor{white}{\textbf{\cellcolor{airforceblue}Beschreibung}} &  
		\cellcolor{testblau}{Dieser Test pr�ft, ob die Registrierung mit einem validen User funktioniert }
		\tabularnewline
		\tabucline[2pt]{-}
		\textcolor{white}{\textbf{\cellcolor{airforceblue}Version}} &
		\cellcolor{testblau} 1.0
		\tabularnewline
		\tabucline[2pt]{-}
		\textcolor{white}{\textbf{\cellcolor{airforceblue}Getestete Anforderung}} &
		\cellcolor{testblau} REQ0.02 --- Registrierung eines Users mit dem System
		\tabularnewline
		\tabucline[2pt]{-}
		\textcolor{white}{\textbf{\cellcolor{airforceblue}Testschritte }} &
		\cellcolor{testblau}
		\begin{enumerate}
		\itemsep -0.4em
			\item Navigieren zur Registrierungsseite (webserver/signup)
			\item Einen neuen validen User registrieren

		\end{enumerate}
		\tabularnewline
		\tabucline[2pt]{-}
		\textcolor{white}{\textbf{\cellcolor{airforceblue}Erwartetes Resultat}} &
		\cellcolor{testblau}	
		Erfolgreiche Registrierung, Weiterleitung zur Profil- oder Startseite.
	\end{longtabu}
\end{center}
\begin{center}
	\tabulinesep = 1mm
	\begin{longtabu} to \linewidth [m]{|[2pt]p{3cm}|[2pt]X[1, m , l]|[2pt]}
		\arrayrulecolor{white}	
		\tabucline[2pt]{-}
		\textcolor{white}{\textbf{\cellcolor{airforceblue}TE0.05}}  &   
		\textcolor{white}{\textbf{\cellcolor{airforceblue}Gleiche E-Mail bei Registrierung}}
		\tabularnewline
		\tabucline[2pt]{-}
		\textcolor{white}{\textbf{\cellcolor{airforceblue}Beschreibung}} &  
		\cellcolor{testblau}{Dieser Test pr�ft, ob die Registrierung mit einem schon existierenden User  einen Fehler gibt }
		\tabularnewline
		\tabucline[2pt]{-}
		\textcolor{white}{\textbf{\cellcolor{airforceblue}Version}} &
		\cellcolor{testblau} 1.0
		\tabularnewline
		\tabucline[2pt]{-}
		\textcolor{white}{\textbf{\cellcolor{airforceblue}Getestete Anforderung}} &
		\cellcolor{testblau} REQ0.02 - Registrierung eines Users mit dem System
		\tabularnewline
		\tabucline[2pt]{-}
		\textcolor{white}{\textbf{\cellcolor{airforceblue}Testschritte }} &
		\cellcolor{testblau}
		\begin{enumerate}
		\itemsep -0.4em
			\item Navigieren zur Registrierungsseite (webserver/signup)
			\item Einen existierenden User mit gleicher E-Mail registrieren
		\end{enumerate}
		\tabularnewline
		\tabucline[2pt]{-}
		\textcolor{white}{\textbf{\cellcolor{airforceblue}Erwartetes Resultat}} &
		\cellcolor{testblau}	
		Fehlermeldung, der User ist vorhanden
	\end{longtabu}
\end{center}
\begin{center}
	\tabulinesep = 1mm
	\begin{longtabu} to \linewidth [m]{|[2pt]p{3cm}|[2pt]X[1, m , l]|[2pt]}
		\arrayrulecolor{white}	
		\tabucline[2pt]{-}
		\textcolor{white}{\textbf{\cellcolor{airforceblue}TE1.01}}  &   
		\textcolor{white}{\textbf{\cellcolor{airforceblue}User registriert sich und verl�sst  Racket-Sportzentrum}}
		\tabularnewline
		\tabucline[2pt]{-}
		\textcolor{white}{\textbf{\cellcolor{airforceblue}Beschreibung}} &  
		\cellcolor{testblau}{Der User w�hlt ein Racket-Sportzentrum aus und registriert sich darin }
		\tabularnewline
		\tabucline[2pt]{-}
		\textcolor{white}{\textbf{\cellcolor{airforceblue}Version}} &
		\cellcolor{testblau} 1.0
		\tabularnewline
		\tabucline[2pt]{-}
		\textcolor{white}{\textbf{\cellcolor{airforceblue}Getestete Anforderung}} &
		\cellcolor{testblau} REQ0.02 --- Registrierung eines Users mit dem System
		\tabularnewline
		\tabucline[2pt]{-}
		\textcolor{white}{\textbf{\cellcolor{airforceblue}Testschritte }} &
		\cellcolor{testblau}
		\begin{enumerate}
		\itemsep -0.4em
			\item Navigieren zur Courtsite (webserver/courts)
			\item Ein Racket-Sportzentrum ausw�hlen.
			\item Auf  \guillemotleft Ich will mitspielen\guillemotright {} dr�cken, um sich mit dem Racket-Sportzentrum zu registrieren.
			\item Auf \guillemotleft Ich will nicht mehr mitspielen\guillemotright {}  dr�cken, um sich auszutragen.
		\end{enumerate}
		\tabularnewline
		\tabucline[2pt]{-}
		\textcolor{white}{\textbf{\cellcolor{airforceblue}Erwartetes Resultat}} &
		\cellcolor{testblau}	
		Nach Schritt 3 muss der User in der Liste des Racket-Sportzentrums auftauchen. Nach Schritt 4 muss der User aus der Liste verschwinden
	\end{longtabu}
\end{center}
\begin{center}
	\tabulinesep = 1mm
	\begin{longtabu} to \linewidth [m]{|[2pt]p{3cm}|[2pt]X[1, m , l]|[2pt]}
		\arrayrulecolor{white}	
		\tabucline[2pt]{-}
		\textcolor{white}{\textbf{\cellcolor{airforceblue}TE1.02}}  &   
		\textcolor{white}{\textbf{\cellcolor{airforceblue}Details zu Spielern}}
		\tabularnewline
		\tabucline[2pt]{-}
		\textcolor{white}{\textbf{\cellcolor{airforceblue}Beschreibung}} &  
		\cellcolor{testblau}{Der Spieler kann Details �ber sich preisgeben. Es gibt eine Seite, auf der er die Informationen �ber ein Formular eintragen kann.}
		\tabularnewline
		\tabucline[2pt]{-}
		\textcolor{white}{\textbf{\cellcolor{airforceblue}Version}} &
		\cellcolor{testblau} 1.0
		\tabularnewline
		\tabucline[2pt]{-}
		\textcolor{white}{\textbf{\cellcolor{airforceblue}Getestete Anforderung}} &
		\cellcolor{testblau} REQ1.01 --- Zuteilung von User zu Racket-Sportzentren
		\tabularnewline
		\tabucline[2pt]{-}
		\textcolor{white}{\textbf{\cellcolor{airforceblue}Testschritte }} &
		\cellcolor{testblau}
		\begin{enumerate}
		\itemsep -0.4em
			\item Navigieren zur Profilseite (webserver/profile)
			\item Informationen �ber das Formular eintragen
			\item Formular abschicken
		
		\end{enumerate}
		\tabularnewline
		\tabucline[2pt]{-}
		\textcolor{white}{\textbf{\cellcolor{airforceblue}Erwartetes Resultat}} &
		\cellcolor{testblau}	
		Informationen nach einem Reload immer noch auf der Profilseite vorhanden.
	\end{longtabu}
\end{center}
\begin{center}
	\tabulinesep = 1mm
	\begin{longtabu} to \linewidth [m]{|[2pt]p{3cm}|[2pt]X[1, m , l]|[2pt]}
		\arrayrulecolor{white}	
		\tabucline[2pt]{-}
		\textcolor{white}{\textbf{\cellcolor{airforceblue}TE1.03}}  &   
		\textcolor{white}{\textbf{\cellcolor{airforceblue}Friend System - Freund einladen}}
		\tabularnewline
		\tabucline[2pt]{-}
		\textcolor{white}{\textbf{\cellcolor{airforceblue}Beschreibung}} &  
		\cellcolor{testblau}{Spieler 1 l�dt Spieler 2 ein, ein Freund zu werden. Spieler 2 sieht die Einladung und kann sie annehmen oder ablehnen.}
		\tabularnewline
		\tabucline[2pt]{-}
		\textcolor{white}{\textbf{\cellcolor{airforceblue}Version}} &
		\cellcolor{testblau} 1.0
		\tabularnewline
		\tabucline[2pt]{-}
		\textcolor{white}{\textbf{\cellcolor{airforceblue}Getestete Anforderung}} &
		\cellcolor{testblau} REQ1.03 --- Friend System
		\tabularnewline
		\tabucline[2pt]{-}
		\textcolor{white}{\textbf{\cellcolor{airforceblue}Testschritte }} &
		\cellcolor{testblau}
		\begin{enumerate}
		\itemsep -0.4em
			\item Navigieren zur Freundeseite (webserver/friends)
			\item Name von Spieler 2 in Formular eingeben
			\item Einladung senden
		\end{enumerate}
		\tabularnewline
		\tabucline[2pt]{-}
		\textcolor{white}{\textbf{\cellcolor{airforceblue}Erwartetes Resultat}} &
		\cellcolor{testblau}	
		Spieler 2 sieht eine Einladung von Spieler 1
	\end{longtabu}
\end{center}
\begin{center}
	\tabulinesep = 1mm
	\begin{longtabu} to \linewidth [m]{|[2pt]p{3cm}|[2pt]X[1, m , l]|[2pt]}
		\arrayrulecolor{white}	
		\tabucline[2pt]{-}
		\textcolor{white}{\textbf{\cellcolor{airforceblue}TE1.04}}  &   
		\textcolor{white}{\textbf{\cellcolor{airforceblue}Friend System --- Einladung annehmen}}
		\tabularnewline
		\tabucline[2pt]{-}
		\textcolor{white}{\textbf{\cellcolor{airforceblue}Beschreibung}} &  
		\cellcolor{testblau}{Spieler 2 nimmt die Einladung an. Beide Spieler sehen sich nun als Freunde}
		\tabularnewline
		\tabucline[2pt]{-}
		\textcolor{white}{\textbf{\cellcolor{airforceblue}Version}} &
		\cellcolor{testblau} 1.0
		\tabularnewline
		\tabucline[2pt]{-}
		\textcolor{white}{\textbf{\cellcolor{airforceblue}Getestete Anforderung}} &
		\cellcolor{testblau} REQ1.03 --- Friend System
		\tabularnewline
		\tabucline[2pt]{-}
		\textcolor{white}{\textbf{\cellcolor{airforceblue}Testschritte }} &
		\cellcolor{testblau}
		\begin{enumerate}
		\itemsep -0.4em
			\item Navigieren zur Freundeseite (webserver/friends)
			\item Einladung von Spieler 1 annehmen
		\end{enumerate}
		\tabularnewline
		\tabucline[2pt]{-}
		\textcolor{white}{\textbf{\cellcolor{airforceblue}Erwartetes Resultat}} &
		\cellcolor{testblau}	
		Beide sehen sich als Freunde auf der Freundeseite.
	\end{longtabu}
\end{center}
\begin{center}
	\tabulinesep = 1mm
	\begin{longtabu} to \linewidth [m]{|[2pt]p{3cm}|[2pt]X[1, m , l]|[2pt]}
		\arrayrulecolor{white}	
		\tabucline[2pt]{-}
		\textcolor{white}{\textbf{\cellcolor{airforceblue}TE1.05}}  &   
		\textcolor{white}{\textbf{\cellcolor{airforceblue}Friend System --- Einladung ablehnen}}
		\tabularnewline
		\tabucline[2pt]{-}
		\textcolor{white}{\textbf{\cellcolor{airforceblue}Beschreibung}} &  
		\cellcolor{testblau}{Spieler 2 lehnt die Einladung ab. Beide Spieler sehen sich nicht als freunde}
		\tabularnewline
		\tabucline[2pt]{-}
		\textcolor{white}{\textbf{\cellcolor{airforceblue}Version}} &
		\cellcolor{testblau} 1.0
		\tabularnewline
		\tabucline[2pt]{-}
		\textcolor{white}{\textbf{\cellcolor{airforceblue}Getestete Anforderung}} &
		\cellcolor{testblau} REQ1.03 --- Friend System
		\tabularnewline
		\tabucline[2pt]{-}
		\textcolor{white}{\textbf{\cellcolor{airforceblue}Testschritte }} &
		\cellcolor{testblau}
		\begin{enumerate}
		\itemsep -0.4em
			\item Navigieren zur Freundeseite (webserver/friends)
			\item Einladung von Spieler 1 ablehnen
		\end{enumerate}
		\tabularnewline
		\tabucline[2pt]{-}
		\textcolor{white}{\textbf{\cellcolor{airforceblue}Erwartetes Resultat}} &
		\cellcolor{testblau}	
		Beide sehen sich nicht als Freunde auf der Freundeseite.
	\end{longtabu}
\end{center}
\begin{center}
	\tabulinesep = 1mm
	\begin{longtabu} to \linewidth [m]{|[2pt]p{3cm}|[2pt]X[1, m , l]|[2pt]}
		\arrayrulecolor{white}	
		\tabucline[2pt]{-}
		\textcolor{white}{\textbf{\cellcolor{airforceblue}TE1.06}}  &   
		\textcolor{white}{\textbf{\cellcolor{airforceblue}Friend System --- nicht vorhandenen User einladen}}
		\tabularnewline
		\tabucline[2pt]{-}
		\textcolor{white}{\textbf{\cellcolor{airforceblue}Beschreibung}} &  
		\cellcolor{testblau}{Spieler 1 l�dt einen nicht vorhandenen User ein.}
		\tabularnewline
		\tabucline[2pt]{-}
		\textcolor{white}{\textbf{\cellcolor{airforceblue}Version}} &
		\cellcolor{testblau} 1.0
		\tabularnewline
		\tabucline[2pt]{-}
		\textcolor{white}{\textbf{\cellcolor{airforceblue}Getestete Anforderung}} &
		\cellcolor{testblau} REQ1.03 --- Friend System
		\tabularnewline
		\tabucline[2pt]{-}
		\textcolor{white}{\textbf{\cellcolor{airforceblue}Testschritte }} &
		\cellcolor{testblau}
		\begin{enumerate}
		\itemsep -0.4em
			\item Navigieren zur Freundeseite (webserver/friends)
			\item Einladung an Username senden, welcher nicht existiert
		\end{enumerate}
		\tabularnewline
		\tabucline[2pt]{-}
		\textcolor{white}{\textbf{\cellcolor{airforceblue}Erwartetes Resultat}} &
		\cellcolor{testblau}	
		Ein Fehler wird zur�ckgegeben, dass der User nicht gefunden wurde.
	\end{longtabu}
\end{center}
\begin{center}
	\tabulinesep = 1mm
	\begin{longtabu} to \linewidth [m]{|[2pt]p{3cm}|[2pt]X[1, m , l]|[2pt]}
		\arrayrulecolor{white}	
		\tabucline[2pt]{-}
		\textcolor{white}{\textbf{\cellcolor{airforceblue}TE2.01}}  &   
		\textcolor{white}{\textbf{\cellcolor{airforceblue}Spiel mit Terminvorschlag erstellen}}
		\tabularnewline
		\tabucline[2pt]{-}
		\textcolor{white}{\textbf{\cellcolor{airforceblue}Beschreibung}} &  
		\cellcolor{testblau}{Ein Spieler erstellt ein Spiel.}
		\tabularnewline
		\tabucline[2pt]{-}
		\textcolor{white}{\textbf{\cellcolor{airforceblue}Version}} &
		\cellcolor{testblau} 1.0
		\tabularnewline
		\tabucline[2pt]{-}
		\textcolor{white}{\textbf{\cellcolor{airforceblue}Getestete Anforderung}} &
		\cellcolor{testblau} 
		\begin{itemize}
		\item REQ2.01 --- Spiel erstellen
		\item REQ2.02 --- Spiel Terminvorschl�ge erstellen
		\end{itemize}
		\tabularnewline
		\tabucline[2pt]{-}
		\textcolor{white}{\textbf{\cellcolor{airforceblue}Testschritte }} &
		\cellcolor{testblau}
		\begin{enumerate}
		\itemsep -0.4em
			\item Navigieren zur Spieleseite (webserver/matches)
			\item Erstellen eines Spiels
			\item Alle Felder einf�llen , inklusive Terminvorschl�ge
			\item Spiel abschicken
		\end{enumerate}
		\tabularnewline
		\tabucline[2pt]{-}
		\textcolor{white}{\textbf{\cellcolor{airforceblue}Erwartetes Resultat}} &
		\cellcolor{testblau}	
		Spieler 2 sollte nun das Spiel sehen
	\end{longtabu}
\end{center}
\begin{center}
	\tabulinesep = 1mm
	\begin{longtabu} to \linewidth [m]{|[2pt]p{3cm}|[2pt]X[1, m , l]|[2pt]}
		\arrayrulecolor{white}	
		\tabucline[2pt]{-}
		\textcolor{white}{\textbf{\cellcolor{airforceblue}TE2.02}}  &   
		\textcolor{white}{\textbf{\cellcolor{airforceblue}Nicht valides Spiel erstellen}}
		\tabularnewline
		\tabucline[2pt]{-}
		\textcolor{white}{\textbf{\cellcolor{airforceblue}Beschreibung}} &  
		\cellcolor{testblau}{Ein Spieler erstellt ein Spiel}
		\tabularnewline
		\tabucline[2pt]{-}
		\textcolor{white}{\textbf{\cellcolor{airforceblue}Version}} &
		\cellcolor{testblau} 1.0
		\tabularnewline
		\tabucline[2pt]{-}
		\textcolor{white}{\textbf{\cellcolor{airforceblue}Getestete Anforderung}} &
		\cellcolor{testblau} 
		\begin{itemize}
		\item REQ2.01 --- Spiel erstellen
		\item REQ2.02 --- Spiel Terminvorschl�ge erstellen
		\end{itemize}
		\tabularnewline
		\tabucline[2pt]{-}
		\textcolor{white}{\textbf{\cellcolor{airforceblue}Testschritte }} &
		\cellcolor{testblau}
		\begin{enumerate}
		\itemsep -0.4em
			\item Navigieren zur Spieleseite (webserver/matches)
			\item Erstellen eines Spiels
			\item Kein Sport ausf�llen
			\item Spiel abschicken
		\end{enumerate}
		\tabularnewline
		\tabucline[2pt]{-}
		\textcolor{white}{\textbf{\cellcolor{airforceblue}Erwartetes Resultat}} &
		\cellcolor{testblau}	
		Beim abschicken sollte ein Fehler erscheinen, dass das Spiel nicht vollst�ndig ausgef�llt wurde.
	\end{longtabu}
\end{center}

\begin{center}
	\tabulinesep = 1mm
	\begin{longtabu} to \linewidth [m]{|[2pt]p{3cm}|[2pt]X[1, m , l]|[2pt]}
		\arrayrulecolor{white}	
		\tabucline[2pt]{-}
		\textcolor{white}{\textbf{\cellcolor{airforceblue}TE2.03}}  &   
		\textcolor{white}{\textbf{\cellcolor{airforceblue}Spiel Terminvorschlag annehmen}}
		\tabularnewline
		\tabucline[2pt]{-}
		\textcolor{white}{\textbf{\cellcolor{airforceblue}Beschreibung}} &  
		\cellcolor{testblau}{Ein Spieler nimmt ein Spiel an.}
		\tabularnewline
		\tabucline[2pt]{-}
		\textcolor{white}{\textbf{\cellcolor{airforceblue}Version}} &
		\cellcolor{testblau} 1.0
		\tabularnewline
		\tabucline[2pt]{-}
		\textcolor{white}{\textbf{\cellcolor{airforceblue}Getestete Anforderung}} &
		\cellcolor{testblau} 
		 REQ2.03 --- Spiel Terminvorschl�ge annehmen und ablehnen
		\tabularnewline
		\tabucline[2pt]{-}
		\textcolor{white}{\textbf{\cellcolor{airforceblue}Testschritte }} &
		\cellcolor{testblau}
		\begin{enumerate}
		\itemsep -0.4em
			\item Navigieren zur Spieleseite (webserver/matches)
			\item Einen Spielvorschlag ausw�hlen
			\item Einen Terminvorschlag w�hlen
			\item Spiel abschicken
		\end{enumerate}
		\tabularnewline
		\tabucline[2pt]{-}
		\textcolor{white}{\textbf{\cellcolor{airforceblue}Erwartetes Resultat}} &
		\cellcolor{testblau}	
		Der User sieht das Spiel nicht mehr, der andere Spieler kann nun den Vorschlag best�tigen
	\end{longtabu}
\end{center}
\begin{center}
	\tabulinesep = 1mm
	\begin{longtabu} to \linewidth [m]{|[2pt]p{3cm}|[2pt]X[1, m , l]|[2pt]}
		\arrayrulecolor{white}	
		\tabucline[2pt]{-}
		\textcolor{white}{\textbf{\cellcolor{airforceblue}TE2.04}}  &   
		\textcolor{white}{\textbf{\cellcolor{airforceblue}Spiel Terminvorschlag ablehnen}}
		\tabularnewline
		\tabucline[2pt]{-}
		\textcolor{white}{\textbf{\cellcolor{airforceblue}Beschreibung}} &  
		\cellcolor{testblau}{Ein Spieler nimmt ein Spiel an.}
		\tabularnewline
		\tabucline[2pt]{-}
		\textcolor{white}{\textbf{\cellcolor{airforceblue}Version}} &
		\cellcolor{testblau} 1.0
		\tabularnewline
		\tabucline[2pt]{-}
		\textcolor{white}{\textbf{\cellcolor{airforceblue}Getestete Anforderung}} &
		\cellcolor{testblau} 
		 REQ2.03 --- Spiel Terminvorschl�ge annehmen und ablehnen
		\tabularnewline
		\tabucline[2pt]{-}
		\textcolor{white}{\textbf{\cellcolor{airforceblue}Testschritte }} &
		\cellcolor{testblau}
		\begin{enumerate}
		\itemsep -0.4em
			\item Navigieren zur Spieleseite (webserver/matches)
			\item Einen Spielvorschlag ausw�hlen
			\item Neue Termine ausw�hlen dr�cken
			\item Neue Termine ausw�hlen
			\item Spiel zur�ckschicken
		\end{enumerate}
		\tabularnewline
		\tabucline[2pt]{-}
		\textcolor{white}{\textbf{\cellcolor{airforceblue}Erwartetes Resultat}} &
		\cellcolor{testblau}	
		Das Spiel geht zur�ck zum anderen Spieler und dieser kann nun von den neuen Vorschl�gen ausw�hlen.
	\end{longtabu}
\end{center}
\begin{center}
	\tabulinesep = 1mm
	\begin{longtabu} to \linewidth [m]{|[2pt]p{3cm}|[2pt]X[1, m , l]|[2pt]}
		\arrayrulecolor{white}	
		\tabucline[2pt]{-}
		\textcolor{white}{\textbf{\cellcolor{airforceblue}TE2.05}}  &   
		\textcolor{white}{\textbf{\cellcolor{airforceblue}Broadcast Spiel erstellen}}
		\tabularnewline
		\tabucline[2pt]{-}
		\textcolor{white}{\textbf{\cellcolor{airforceblue}Beschreibung}} &  
		\cellcolor{testblau}{Der Spieler erstellt ein Broadcast Spiel f�r das Racket-Sportzentrum}
		\tabularnewline
		\tabucline[2pt]{-}
		\textcolor{white}{\textbf{\cellcolor{airforceblue}Version}} &
		\cellcolor{testblau} 1.0
		\tabularnewline
		\tabucline[2pt]{-}
		\textcolor{white}{\textbf{\cellcolor{airforceblue}Getestete Anforderung}} &
		\cellcolor{testblau} 
		 REQ2.05 --- Broadcast Spiele erstellen
		\tabularnewline
		\tabucline[2pt]{-}
		\textcolor{white}{\textbf{\cellcolor{airforceblue}Testschritte }} &
		\cellcolor{testblau}
		\begin{enumerate}
		\itemsep -0.4em
			\item Navigieren zur Spieleseite (webserver/matches)
			\item W�hlt die Erstellung eines Broadcast Spiels
			\item F�llt das Formular aus
			\item Sendet Spiel ab
		\end{enumerate}
		\tabularnewline
		\tabucline[2pt]{-}
		\textcolor{white}{\textbf{\cellcolor{airforceblue}Erwartetes Resultat}} &
		\cellcolor{testblau}	
		Spieler im gleichen Racket-Sportzentrum sehen das Spiel.
	\end{longtabu}
\end{center}
\begin{center}
	\tabulinesep = 1mm
	\begin{longtabu} to \linewidth [m]{|[2pt]p{3cm}|[2pt]X[1, m , l]|[2pt]}
		\arrayrulecolor{white}	
		\tabucline[2pt]{-}
		\textcolor{white}{\textbf{\cellcolor{airforceblue}TE2.06}}  &   
		\textcolor{white}{\textbf{\cellcolor{airforceblue}Ergebnisse des Spiels eintragen}}
		\tabularnewline
		\tabucline[2pt]{-}
		\textcolor{white}{\textbf{\cellcolor{airforceblue}Beschreibung}} &  
		\cellcolor{testblau}{Sobald das Spiel beendet wurde, kann das Ergebnis eingetragen werden}
		\tabularnewline
		\tabucline[2pt]{-}
		\textcolor{white}{\textbf{\cellcolor{airforceblue}Version}} &
		\cellcolor{testblau} 1.0
		\tabularnewline
		\tabucline[2pt]{-}
		\textcolor{white}{\textbf{\cellcolor{airforceblue}Getestete Anforderung}} &
		\cellcolor{testblau} 
		 REQ2.07 --- Ergebnisse des Spiels eintragen
		\tabularnewline
		\tabucline[2pt]{-}
		\textcolor{white}{\textbf{\cellcolor{airforceblue}Testschritte }} &
		\cellcolor{testblau}
		\begin{enumerate}
		\itemsep -0.4em
			\item Navigieren zur Spieleseite (webserver/matches)
			\item Ein offenes Spiel w�hlen
			\item Spiel in den Status \guillemotleft In Progress\guillemotright {} setzen
			\item Ergebnisse eintragen
		\end{enumerate}
		\tabularnewline
		\tabucline[2pt]{-}
		\textcolor{white}{\textbf{\cellcolor{airforceblue}Erwartetes Resultat}} &
		\cellcolor{testblau}	
		 Der zweite Spieler sieht das Spiel und kann die Ergebnisse best�tigen
	\end{longtabu}
\end{center}
\begin{center}
	\tabulinesep = 1mm
	\begin{longtabu} to \linewidth [m]{|[2pt]p{3cm}|[2pt]X[1, m , l]|[2pt]}
		\arrayrulecolor{white}	
		\tabucline[2pt]{-}
		\textcolor{white}{\textbf{\cellcolor{airforceblue}TE2.07}}  &   
		\textcolor{white}{\textbf{\cellcolor{airforceblue}Ergebnisse des Spiels eintragen}}
		\tabularnewline
		\tabucline[2pt]{-}
		\textcolor{white}{\textbf{\cellcolor{airforceblue}Beschreibung}} &  
		\cellcolor{testblau}{Sobald das Spiel beendet wurde, kann das Ergebnis eingetragen werden}
		\tabularnewline
		\tabucline[2pt]{-}
		\textcolor{white}{\textbf{\cellcolor{airforceblue}Version}} &
		\cellcolor{testblau} 1.0
		\tabularnewline
		\tabucline[2pt]{-}
		\textcolor{white}{\textbf{\cellcolor{airforceblue}Getestete Anforderung}} &
		\cellcolor{testblau} 
		 REQ2.08 --- Best�tigung des Spiels eintragen
		\tabularnewline
		\tabucline[2pt]{-}
		\textcolor{white}{\textbf{\cellcolor{airforceblue}Testschritte }} &
		\cellcolor{testblau}
		\begin{enumerate}
		\itemsep -0.4em
			\item Navigieren zur Spieleseite (webserver/matches)
			\item Ein Spiel ausw�hlen, welches beendet wurde
			\item Ergebnisse best�tigen
		\end{enumerate}
		\tabularnewline
		\tabucline[2pt]{-}
		\textcolor{white}{\textbf{\cellcolor{airforceblue}Erwartetes Resultat}} &
		\cellcolor{testblau}	
		 Spiel ist nun f�r beide Spieler unter Archiv ersichtlich.
	\end{longtabu}
\end{center}
\begin{center}
	\tabulinesep = 1mm
	\begin{longtabu} to \linewidth [m]{|[2pt]p{3cm}|[2pt]X[1, m , l]|[2pt]}
		\arrayrulecolor{white}	
		\tabucline[2pt]{-}
		\textcolor{white}{\textbf{\cellcolor{airforceblue}TE2.08}}  &   
		\textcolor{white}{\textbf{\cellcolor{airforceblue}Spiel-Workflow End-To-End Test}}
		\tabularnewline
		\tabucline[2pt]{-}
		\textcolor{white}{\textbf{\cellcolor{airforceblue}Beschreibung}} &  
		\cellcolor{testblau}{Zwei Spieler spielen den ganzen Spiel Workflow durch}
		\tabularnewline
		\tabucline[2pt]{-}
		\textcolor{white}{\textbf{\cellcolor{airforceblue}Version}} &
		\cellcolor{testblau} 1.0
		\tabularnewline
		\tabucline[2pt]{-}
		\textcolor{white}{\textbf{\cellcolor{airforceblue}Getestete Anforderung}} &
		\cellcolor{testblau} 
		\begin{tabular}[x]{@{}l@{}}
		 REQ2.01 - Spiel erstellen  \\
		 REQ2.02 - Spiel-Terminvorschl�ge erstellen\\
		 REQ2.03 - Spiel-Terminvorschl�ge annehmen und ablehnen\\
		 REQ2.04 - Wiederkehrende Spiele erstellen\\
		 REQ2.05 - Broadcast Spiele erstellen\\
		 REQ2.06 - Privatsph�reeinstellungen ber�cksichtigen\\
		 REQ2.07 - Ergebnisse des Spiels eintragen\\
		 REQ2.08 - Best�tigung des Spiels eintragen
		 \end{tabular}
		\tabularnewline
		\tabucline[2pt]{-}
		\textcolor{white}{\textbf{\cellcolor{airforceblue}Testschritte }} &
		\cellcolor{testblau}
		\begin{enumerate}
		\itemsep -0.4em
			\item Spieler 1: Erstellt Spiel mit Terminvorschl�gen
			\item Spieler 2: Best�tigt ein Terminvorschlag
			\item Spieler 1: Best�tigt den Terminvorschlag
			\item Spieler 1: Er�ffnet Spiel
			\item Spieler 2: Tr�gt Resultat ein
			\item Spieler 1: Best�tigt Resultat
			\item Spieler 1 und 2: Findet Spiel im Archiv
		\end{enumerate}
		\tabularnewline
		\tabucline[2pt]{-}
		\textcolor{white}{\textbf{\cellcolor{airforceblue}Erwartetes Resultat}} &
		\cellcolor{testblau}	
			Beide Spieler finden Spiel im Archiv
	\end{longtabu}
\end{center}

\begin{center}
	\tabulinesep = 1mm
	\begin{longtabu} to \linewidth [m]{|[2pt]p{3cm}|[2pt]X[1, m , l]|[2pt]}
		\arrayrulecolor{white}	
		\tabucline[2pt]{-}
		\textcolor{white}{\textbf{\cellcolor{airforceblue}TE3.01}}  &   
		\textcolor{white}{\textbf{\cellcolor{airforceblue}Erfolgreich einen Court erstellen}}
		\tabularnewline
		\tabucline[2pt]{-}
		\textcolor{white}{\textbf{\cellcolor{airforceblue}Beschreibung}} &  
		\cellcolor{testblau}{Sobald das Spiel beendet wurde, kann das Ergebnis eingetragen werden}
		\tabularnewline
		\tabucline[2pt]{-}
		\textcolor{white}{\textbf{\cellcolor{airforceblue}Version}} &
		\cellcolor{testblau} 1.0
		\tabularnewline
		\tabucline[2pt]{-}
		\textcolor{white}{\textbf{\cellcolor{airforceblue}Getestete Anforderung}} &
		\cellcolor{testblau} 
		 REQ3.01 --- Courts erstellen
		\tabularnewline
		\tabucline[2pt]{-}
		\textcolor{white}{\textbf{\cellcolor{airforceblue}Testschritte }} &
		\cellcolor{testblau}
		\begin{enumerate}
		\itemsep -0.4em
			\item Navigieren zur Courtseite (webserver/courts)
			\item Einen neuen Court erstellen
		\end{enumerate}
		\tabularnewline
		\tabucline[2pt]{-}
		\textcolor{white}{\textbf{\cellcolor{airforceblue}Erwartetes Resultat}} &
		\cellcolor{testblau}	
		Die eingegebenen Details sind in der Court Detail Seite ersichtlich.
	\end{longtabu}
\end{center}
\begin{center}
	\tabulinesep = 1mm
	\begin{longtabu} to \linewidth [m]{|[2pt]p{3cm}|[2pt]X[1, m , l]|[2pt]}
		\arrayrulecolor{white}	
		\tabucline[2pt]{-}
		\textcolor{white}{\textbf{\cellcolor{airforceblue}TE3.02}}  &   
		\textcolor{white}{\textbf{\cellcolor{airforceblue}Erfolglos einen Court erstellen}}
		\tabularnewline
		\tabucline[2pt]{-}
		\textcolor{white}{\textbf{\cellcolor{airforceblue}Beschreibung}} &  
		\cellcolor{testblau}{Koordinaten f�r den Court sind obligatorisch, wenn eine Koordinate fehlt, den Court sollte nicht erstellt werden, daf�r ein Fehler kommen. }
		\tabularnewline
		\tabucline[2pt]{-}
		\textcolor{white}{\textbf{\cellcolor{airforceblue}Version}} &
		\cellcolor{testblau} 1.0
		\tabularnewline
		\tabucline[2pt]{-}
		\textcolor{white}{\textbf{\cellcolor{airforceblue}Getestete Anforderung}} &
		\cellcolor{testblau} 
		 REQ3.01 --- Courts erstellen
		\tabularnewline
		\tabucline[2pt]{-}
		\textcolor{white}{\textbf{\cellcolor{airforceblue}Testschritte }} &
		\cellcolor{testblau}
		\begin{enumerate}
		\itemsep -0.4em
			\item Navigieren zur Courtseite (webserver/courts)
			\item Einen neuen Court erstellen
			\item Adresse nicht ausf�llen
		\end{enumerate}
		\tabularnewline
		\tabucline[2pt]{-}
		\textcolor{white}{\textbf{\cellcolor{airforceblue}Erwartetes Resultat}} &
		\cellcolor{testblau}	
		Ein Fehler sollte erscheinen.
	\end{longtabu}
\end{center}

\begin{center}
	\tabulinesep = 1mm
	\begin{longtabu} to \linewidth [m]{|[2pt]p{3cm}|[2pt]X[1, m , l]|[2pt]}
		\arrayrulecolor{white}	
		\tabucline[2pt]{-}
		\textcolor{white}{\textbf{\cellcolor{airforceblue}TE3.03}}  &   
		\textcolor{white}{\textbf{\cellcolor{airforceblue}Spieler k�nnen sich im Court eintragen}}
		\tabularnewline
		\tabucline[2pt]{-}
		\textcolor{white}{\textbf{\cellcolor{airforceblue}Beschreibung}} &  
		\cellcolor{testblau}{Spieler kann sich bei Court registrieren, und anschliessend die Registration annullieren }
		\tabularnewline
		\tabucline[2pt]{-}
		\textcolor{white}{\textbf{\cellcolor{airforceblue}Version}} &
		\cellcolor{testblau} 1.0
		\tabularnewline
		\tabucline[2pt]{-}
		\textcolor{white}{\textbf{\cellcolor{airforceblue}Getestete Anforderung}} &
		\cellcolor{testblau} 
		 REQ3.02 --- Spieler k�nnen sich in Courts eintragen
		\tabularnewline
		\tabucline[2pt]{-}
		\textcolor{white}{\textbf{\cellcolor{airforceblue}Testschritte }} &
		\cellcolor{testblau}
		\begin{enumerate}
		\itemsep -0.4em
			\item Navigieren zur Courtseite (webserver/courts)
			\item Courtdetails �ffnen
			\item Sich in Court einschreiben
			\item Sich in Court ausschreiben
		\end{enumerate}
		\tabularnewline
		\tabucline[2pt]{-}
		\textcolor{white}{\textbf{\cellcolor{airforceblue}Erwartetes Resultat}} &
		\cellcolor{testblau}	
		Nach dem 3. Schritt sollte der User in der Liste der Spieler vom Court auftauchen, nach Schritt 4 sollte er wieder verschwunden sein.
	\end{longtabu}
	\end{center}
\begin{center}
	\tabulinesep = 1mm
	\begin{longtabu} to \linewidth [m]{|[2pt]p{3cm}|[2pt]X[1, m , l]|[2pt]}
		\arrayrulecolor{white}	
		\tabucline[2pt]{-}
		\textcolor{white}{\textbf{\cellcolor{airforceblue}TE4.01}}  &   
		\textcolor{white}{\textbf{\cellcolor{airforceblue}Erfolgreich eine Liga erstellen}}
		\tabularnewline
		\tabucline[2pt]{-}
		\textcolor{white}{\textbf{\cellcolor{airforceblue}Beschreibung}} &  
		\cellcolor{testblau}{Der Spieler kann eine eigene Liga erstellen bei Eingabe von validen Informationen in das Formular }
		\tabularnewline
		\tabucline[2pt]{-}
		\textcolor{white}{\textbf{\cellcolor{airforceblue}Version}} &
		\cellcolor{testblau} 1.0
		\tabularnewline
		\tabucline[2pt]{-}
		\textcolor{white}{\textbf{\cellcolor{airforceblue}Getestete Anforderung}} &
		\cellcolor{testblau} 
		 REQ4.01 --- Liga erstellen
		\tabularnewline
		\tabucline[2pt]{-}
		\textcolor{white}{\textbf{\cellcolor{airforceblue}Testschritte }} &
		\cellcolor{testblau}
		\begin{enumerate}
		\itemsep -0.4em
			\item Navigieren zur Liga (webserver/leagues)
			\item Neue Liga erstellen
			\item Formular ausf�llen
		\end{enumerate}
		\tabularnewline
		\tabucline[2pt]{-}
		\textcolor{white}{\textbf{\cellcolor{airforceblue}Erwartetes Resultat}} &
		\cellcolor{testblau}	
		Liga sollte nun in der �bersicht ersichtlich sein.
	\end{longtabu}
	\end{center}
	\begin{center}
		\tabulinesep = 1mm
		\begin{longtabu} to \linewidth [m]{|[2pt]p{3cm}|[2pt]X[1, m , l]|[2pt]}
			\arrayrulecolor{white}	
			\tabucline[2pt]{-}
			\textcolor{white}{\textbf{\cellcolor{airforceblue}TE4.02}}  &   
			\textcolor{white}{\textbf{\cellcolor{airforceblue}Erfolglos eine Liga erstellen}}
			\tabularnewline
			\tabucline[2pt]{-}
			\textcolor{white}{\textbf{\cellcolor{airforceblue}Beschreibung}} &  
			\cellcolor{testblau}{Der Spieler kann eine eigene Liga erstellen bei Eingabe von validen Informationen in das Formular }
			\tabularnewline
			\tabucline[2pt]{-}
			\textcolor{white}{\textbf{\cellcolor{airforceblue}Version}} &
			\cellcolor{testblau} 1.0
			\tabularnewline
			\tabucline[2pt]{-}
			\textcolor{white}{\textbf{\cellcolor{airforceblue}Getestete Anforderung}} &
			\cellcolor{testblau} 
			 REQ4.01 --- Liga erstellen
			\tabularnewline
			\tabucline[2pt]{-}
			\textcolor{white}{\textbf{\cellcolor{airforceblue}Testschritte }} &
			\cellcolor{testblau}
			\begin{enumerate}
			\itemsep -0.4em
				\item Navigieren zur Ligaseite (webserver/leagues)
				\item Neuen Court erstellen
				\item Sportart nicht ausf�llen

			\end{enumerate}
			\tabularnewline
			\tabucline[2pt]{-}
			\textcolor{white}{\textbf{\cellcolor{airforceblue}Erwartetes Resultat}} &
			\cellcolor{testblau}	
			Ein Fehler sollte auftauchen der verlangt, dass das Feld ausgef�llt wird.
		\end{longtabu}
		\end{center}
	\begin{center}
		\tabulinesep = 1mm
		\begin{longtabu} to \linewidth [m]{|[2pt]p{3cm}|[2pt]X[1, m , l]|[2pt]}
			\arrayrulecolor{white}	
			\tabucline[2pt]{-}
			\textcolor{white}{\textbf{\cellcolor{airforceblue}TE4.03}}  &   
			\textcolor{white}{\textbf{\cellcolor{airforceblue}Spieler k�nnen sich in die Liga eintragen}}
			\tabularnewline
			\tabucline[2pt]{-}
			\textcolor{white}{\textbf{\cellcolor{airforceblue}Beschreibung}} &  
			\cellcolor{testblau}{Spieler kann sich bei Liga registrieren, und anschliessend ausschrieben  }
			\tabularnewline
			\tabucline[2pt]{-}
			\textcolor{white}{\textbf{\cellcolor{airforceblue}Version}} &
			\cellcolor{testblau} 1.0
			\tabularnewline
			\tabucline[2pt]{-}
			\textcolor{white}{\textbf{\cellcolor{airforceblue}Getestete Anforderung}} &
			\cellcolor{testblau} 
			 REQ4.01 --- Liga erstellen
			\tabularnewline
			\tabucline[2pt]{-}
			\textcolor{white}{\textbf{\cellcolor{airforceblue}Testschritte }} &
			\cellcolor{testblau}
			\begin{enumerate}
			\itemsep -0.4em
			\item Navigieren zur Ligaseite (webserver/league)
			\item Ligadetails �ffnen
			\item Sich in Liga einschreiben
			\item Sich in Liga ausschreiben
			\end{enumerate}
			\tabularnewline
			\tabucline[2pt]{-}
			\textcolor{white}{\textbf{\cellcolor{airforceblue}Erwartetes Resultat}} &
			\cellcolor{testblau}	
			Nach Schritt 3 sollte man in der Liste der Liga sein, nach Schritt 4 nicht mehr.
		\end{longtabu}
		\end{center}
	\begin{center}
		\tabulinesep = 1mm
		\begin{longtabu} to \linewidth [m]{|[2pt]p{3cm}|[2pt]X[1, m , l]|[2pt]}
			\arrayrulecolor{white}	
			\tabucline[2pt]{-}
			\textcolor{white}{\textbf{\cellcolor{airforceblue}TE4.04}}  &   
			\textcolor{white}{\textbf{\cellcolor{airforceblue}Rangliste durch Spiele aktualisieren}}
			\tabularnewline
			\tabucline[2pt]{-}
			\textcolor{white}{\textbf{\cellcolor{airforceblue}Beschreibung}} &  
			\cellcolor{testblau}{Beim Abschluss eines Spiels wird die Rangliste der Liga aktualisiert}
			\tabularnewline
			\tabucline[2pt]{-}
			\textcolor{white}{\textbf{\cellcolor{airforceblue}Version}} &
			\cellcolor{testblau} 1.0
			\tabularnewline
			\tabucline[2pt]{-}
			\textcolor{white}{\textbf{\cellcolor{airforceblue}Getestete Anforderung}} &
			\cellcolor{testblau} 
			 REQ4.01 --- Liga erstellen
			\tabularnewline
			\tabucline[2pt]{-}
			\textcolor{white}{\textbf{\cellcolor{airforceblue}Testschritte }} &
			\cellcolor{testblau}
			\begin{enumerate}
			\itemsep -0.4em
			\item Spiel erstellen und an vorhandene Liga kuppeln (Drop-Down in Formular)
			\item Spielworkflow durchspielen
			\item Spiel abschliessen
			\item Punkte in Rangliste �berpr�fen
			\end{enumerate}
			\tabularnewline
			\tabucline[2pt]{-}
			\textcolor{white}{\textbf{\cellcolor{airforceblue}Erwartetes Resultat}} &
			\cellcolor{testblau}	
			Punkte f�r den Gewinner m�ssen aktualisiert sein.
		\end{longtabu}
		\end{center}
\subsection{Testresultate}
\begin{center}
		\tabulinesep = 1mm
		\begin{longtabu} to \linewidth {p{1.5cm}|[2pt]X[1, m , l]|[2pt]X[1, m , l]|[2pt]p{1.5cm}}
			\arrayrulecolor{white}	
			\tabucline[2pt]{-}
			\textcolor{white}{\textbf{\cellcolor{airforceblue}ID}}  &   
			\textcolor{white}{\textbf{\cellcolor{airforceblue}Erwartetes Resultat}}&
			\textcolor{white}{\textbf{\cellcolor{airforceblue}Tats�chliches Resultat}}&
			\textcolor{white}{\textbf{\cellcolor{airforceblue}(N)OK}}    
			\tabularnewline
			\tabucline[2pt]{-}
			\cellcolor{testblau} TE0.01 &
			\cellcolor{testblau}Erfolgreicher Login, Weiterleitung zur Startseite.&
			\cellcolor{testblau} Erfolgreicher Login, Weiterleitung zur Startseite.&
			\cellcolor{testblau} OK
			\tabularnewline
			\tabucline[2pt]{-}
			\cellcolor{testblau} TE0.02 &
			\cellcolor{testblau}Fehlgeschlagener Login, Fehlermeldung (z.B. Wrong Username/Password)&
			\cellcolor{testblau} Fehlgeschlagener Login, Fehlermeldung : Unknown user or invalid password&
			\cellcolor{testblau} OK
			\tabularnewline
			\tabucline[2pt]{-}
			\cellcolor{testblau} TE0.03 &
			\cellcolor{testblau}Erfolgreicher Login, Weiterleitung zur Startseite.&
			\cellcolor{testblau} Erfolgreicher Login, Weiterleitung zur Startseite.&
			\cellcolor{testblau} OK
			\tabularnewline
			\tabucline[2pt]{-}
			\cellcolor{testblau} TE0.04 &
			\cellcolor{testblau}Erfolgreiche Registrierung, Weiterleitung zur Profil oder Startseite.&
			\cellcolor{testblau}Erfolgreiche Registrierung, Weiterleitung zur Profilseite.&
			\cellcolor{testblau} OK
			\tabularnewline
			\tabucline[2pt]{-}
			\cellcolor{testblau} TE0.05 &
			\cellcolor{testblau}Fehlermeldung, der User ist vorhanden&
			\cellcolor{testblau}Fehlermeldung: Username already exists&
			\cellcolor{testblau} OK
			\tabularnewline
			\tabucline[2pt]{-}
			\cellcolor{testblau} TE01.01 &
			\cellcolor{testblau}Nach Schritt 3 muss der User in der Liste des Racket-Sportzentrums
auftauchen. Nach Schritt 4 muss der User aus der Liste verschwinden&
			\cellcolor{testblau}Nach Schritt 3 ist der User in der liste, nach Schritt 4 nicht mehr.&
			\cellcolor{testblau} OK 
			\tabularnewline
			\tabucline[2pt]{-}
			\cellcolor{testblau} TE01.02 &
			\cellcolor{testblau}Informationen nach einem Reload immer noch auf der Profilseite
vorhanden.&
			\cellcolor{testblau}Informationen nach einem Reload immer noch auf der Profilseite
vorhanden.&
			\cellcolor{testblau} OK 
			\tabularnewline
			\tabucline[2pt]{-}
			\cellcolor{testblau} TE01.03 &
			\cellcolor{testblau}Spieler 2 sieht eine Einladung von Spieler 1&
			\cellcolor{testblau}Spieler 2 sieht eine Einladung von Spieler 1&
			\cellcolor{testblau} OK 
			\tabularnewline
			\tabucline[2pt]{-}
			\cellcolor{testblau} TE01.04 &
			\cellcolor{testblau}Beide sehen sich als Freunde auf der Freundeseite.&
			\cellcolor{testblau}Beide sehen sich als Freunde auf der Freundeseite.&
			\cellcolor{testblau} OK 
			\tabularnewline
			\tabucline[2pt]{-}
			\cellcolor{testblau} TE01.05 &
			\cellcolor{testblau}Beide sehen sich nicht als Freunde auf der Freundeseite.&
			\cellcolor{testblau}Beide sehen sich nicht als Freunde auf der Freundeseite.&
			\cellcolor{testblau} OK
			\tabularnewline
			\tabucline[2pt]{-}
			\cellcolor{testblau} TE01.06 &
			\cellcolor{testblau}Ein Fehler wird zur�ckgegeben, dass der User nicht gefunden wurde.&
			\cellcolor{testblau}Fehler: User not found&
			\cellcolor{testblau} OK
			\tabularnewline
			\tabucline[2pt]{-}
			\cellcolor{testblau} TE02.01 &
			\cellcolor{testblau}Spieler 2 sollte nun das Spiel sehen&
			\cellcolor{testblau}Spieler 2 sieht das Spiel in Herausforderungen&
			\cellcolor{testblau} OK
			\tabularnewline
			\tabucline[2pt]{-}
			\cellcolor{notokay} TE02.02 &
			\cellcolor{notokay}Bei abschicken sollte ein Fehler erscheinen, dass das Spiel nicht
vollst�ndig ausgef�llt wurde.&
			\cellcolor{notokay}Spiel wird abgeschickt&
			\cellcolor{notokay} NOK
			\tabularnewline
			\tabucline[2pt]{-}
			\cellcolor{testblau} TE02.03 &
			\cellcolor{testblau}Der User sieht das Spiel nicht mehr, der andere Spieler kann nun den Vorschlag best�tigen&
			\cellcolor{testblau}Der User sieht das Spiel nicht mehr, der andere Spieler kann nun den Vorschlag best�tigen&
			\cellcolor{testblau} OK
			\tabularnewline
			\tabucline[2pt]{-}
			\cellcolor{testblau} TE02.04 &
			\cellcolor{testblau}Das Spiel geht zur�ck zum anderen Spieler und dieser kann nun von
den neuen Vorschlagen ausw�hlen&
			\cellcolor{testblau}Das Spiel geht zur�ck zum anderen Spieler und dieser kann nun von
den neuen Vorschlagen ausw�hlen&
			\cellcolor{testblau} OK
			\tabularnewline
			\tabucline[2pt]{-}
			\cellcolor{testblau} TE02.05 &
			\cellcolor{testblau}Spieler im gleichen Racket-Sportzentrum sehen das Spiel.&
			\cellcolor{testblau}Spieler im gleichen Racket-Sportzentrum sehen das Spiel.&
			\cellcolor{testblau} OK
			\tabularnewline
			\tabucline[2pt]{-}
			\cellcolor{testblau} TE02.06 &
			\cellcolor{testblau}Der zweite Spieler sieht das Spiel und kann die Ergebnisse best�tigen&
			\cellcolor{testblau}Der zweite Spieler sieht das Spiel und kann die Ergebnisse best�tigen&
			\cellcolor{testblau} OK
			\tabularnewline
			\tabucline[2pt]{-}
			\cellcolor{testblau} TE02.07 &
			\cellcolor{testblau}Spiel ist nun f�r beide Spieler unter Archiv ersichtlich&
			\cellcolor{testblau}Spiel ist nun f�r beide Spieler unter Archiv ersichtlich&
			\cellcolor{testblau} OK
			\tabularnewline
			\tabucline[2pt]{-}
			\cellcolor{testblau} TE02.08 &
			\cellcolor{testblau}Beide Spieler finden Spiel im Archiv&
			\cellcolor{testblau}Beide Spieler finden Spiel im Archiv&
			\cellcolor{testblau} OK
			\tabularnewline
			\tabucline[2pt]{-}
			\cellcolor{testblau} TE03.01 &
			\cellcolor{testblau}Die eingegebenen Details sind in der Court Detail Seite ersichtlich.&
			\cellcolor{testblau}Court Details ersichtlich wie eingegeben.&
			\cellcolor{testblau} OK
			\tabularnewline
			\tabucline[2pt]{-}
			\cellcolor{testblau} TE03.02 &
			\cellcolor{testblau}Ein Fehler sollte erscheinen.&
			\cellcolor{testblau}Fehler: Please fill in a correct address (select it from the dropdown) &
			\cellcolor{testblau} OK
			\tabularnewline
			\tabucline[2pt]{-}
			\cellcolor{testblau} TE03.03 &
			\cellcolor{testblau}Nach dem 3. Schritt sollte der User in der Liste der Spieler vom Court auftauchen, nach Schritt 4 sollte er wieder verschwunden sein.&
			\cellcolor{testblau}Nach Schritt 3 wird der User angezeigt, nach Schritt 4 ist er wieder verschwunden. &
			\cellcolor{testblau} OK
			\tabularnewline
			\tabucline[2pt]{-}
			\cellcolor{testblau} TE04.01 &
			\cellcolor{testblau}Liga sollte nun in der �bersicht ersichtlich sein.&
			\cellcolor{testblau}Liga ist in der �bersicht ersichtlich &
			\cellcolor{testblau} OK
			\tabularnewline
			\tabucline[2pt]{-}
			\cellcolor{testblau} TE04.02 &
			\cellcolor{testblau}Ein Fehler sollte auftauchen der verlangt, dass das Feld ausgef�llt wird.&
			\cellcolor{testblau}Fehler: Please fill out the sport type &
			\cellcolor{testblau} OK
			\tabularnewline
			\tabucline[2pt]{-}
			\cellcolor{testblau} TE04.03 &
			\cellcolor{testblau}Nach Schritt 3 sollte man in der Liste der Liga sein, nach Schritt 4 nicht mehr.&
			\cellcolor{testblau}Nach Schritt 3 ist der Spieler in der Liga registriert, nach Schritt 4 nicht mehr&
			\cellcolor{testblau} OK
			\tabularnewline
			\tabucline[2pt]{-}
			\cellcolor{testblau} TE04.04 &
			\cellcolor{testblau}Punkte f�r den Gewinner m�ssen aktualisiert sein.&
			\cellcolor{testblau}Punkte f�r den Gewinner m�ssen aktualisiert sein.&
			\cellcolor{testblau} OK
		\end{longtabu}
		\end{center}
Der Test TE02.02 wurde nicht erfolgreich abgeschlossen. Analysen haben ergeben, dass das Feld Sport nicht als erforderlich getagt wurde. Dies wurde ge�ndert. Der Test ist nun erfolgreich.
% !TeX spellcheck = de_CH
\chapter{Reflektion}
\end{normalsize}

\listoffigures

\printbibliography
\end{spacing}
\end{document}
