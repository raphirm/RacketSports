% !TeX spellcheck = de_CH


\chapter{Inhalt der Arbeit}
Racket-Sportarten haben ein Problem. Man muss die richtige Zeit und den richtigen Partner finden. Die Applikation, welche ich f�r die Semester-Arbeit erstelle, sollte dieses Problem l�sen. Durch einen \textquotedblleft Doodle-Style\textquotedblright-Termin und Spieler-Finder sowie eine Liga-Verwaltung werden die Teilnehmenden motiviert, �fters zu squashen.

\section{Ausgangslage}
Einen Partner f�r Racket-Sportarten zu finden, ist nicht immer sehr einfach. Wenn man schliesslich jemanden gefunden hat, ist es immer schwer, einen Termin zu finden. Spielt man regelm�ssig gegen die gleiche Person, stellt man sich oft auf diese ein und lernt mit der Zeit nur noch wenig dazu. 

Diese App sollte diese Probleme l�sen, indem man einfach miteinander Termine vereinbaren kann und dass Termine vom System f�r die User automatisch generiert werden.

\section{Aufgabenstellung}
Um die Ziele zu erreichen, muss methodisch vorgegangen werden. In der Dokumentation m�ssen angewendete Technologien begr�ndet werden sowie Konzepte und Architekturen der Applikation festgehalten werden.
\begin{enumerate}
\itemsep-0.7em
\item Informieren �ber den momentanen Markt und Organisation rund um Racketsport
\begin{itemize}
	\itemsep-0.7em
	\item Andere Applikationen rund um Racketsport identifizieren
	\item Zielgruppen- und Anwendungsfallanalyse
\end{itemize} 
\item  	Anforderungsanalyse, welche Funktionalit�ten dem Nutzer einen echten Mehrwert bieten
\begin{itemize}
	\itemsep-0.7em
	\item Basierend auf (1) eine Anforderungsanalyse erstellen
	\item User Stories erstellen
	\item Sprints planen
\end{itemize} 
\item Konzeption der App, Technologien w�hlen
\begin{itemize}
	\itemsep-0.7em
	\item Entscheidungsmatrix, welche Technologien verwendet werden
	\item DB Design
	\item Klassen-Diagramm
	\item GUI-Entwurf
\end{itemize}
\item Die Applikation umsetzen
\begin{itemize}
	\itemsep-0.7em
	\item Applikation programmieren 
	\item Unit Tests erstellen/ausf�hren
\end{itemize}
\item Systemtests durchf�hren
\begin{itemize}
	\itemsep-0.7em
	\item Systemstests konzipieren
	\item Systemtests durchf�hren
	\item End-to-End Testing
\end{itemize}
\item Benutzertests durchf�hren
\begin{itemize}
		\itemsep-0.7em
	 \item Benutzer einladen f�r Alpha-Test
	\item Feedback von Benutzer verlangen
\end{itemize}
\end{enumerate}

\section{Erwartete Resultat}
Folgende Resultate werden erwartet:
\begin{itemize}
	\itemsep-0.7em
\item Marktanalyse \& Benutzerverhaltensanalyse zur Vereinbarung von Spielen
\item Anforderungsanalyse
\item Recherche f�r einzusetzende Techniken
\item Konzepte und Implementationstechniken 
\item User Stories und Projektplanung
\item Android App und Webapplikation
\item Testplan und Umsetzung
\item Benutzerfeedback des Alpha-Tests
\item Fazit
\end{itemize}

\section{Zielsetzung der Arbeit}
Ziel der Arbeit, ist eine WebApplikation sowie eine Android App mit bestimmten Funktionalit�ten zu erstellen. Der Autor der Arbeit soll so Zugang zu neuen Technologien im Webbereich erhalten und Erfahrung in der Webapplikations-Programmierung sammeln.

Aufgabenstellung:
\begin{itemize}
	\itemsep-0.7em
\item Dokumentation �ber Entscheidungen und Implementationstechniken
\item Webapplikation mit folgender Funktionalit�t:
\begin{itemize}
	\itemsep-0.7em
	\item RestAPI um alle untenstehenden Funktionalit�ten
	\item Liga erstellen und l�schen
	\item Einer Liga beitreten und eine Liga verlassen
	\item Spiel vereinbaren
	\item Termin f�r Spiel finden
	\item Spielresultat eintragen
	\item Rangliste der Liga berechnen
	\item Automatische Spielvorschl�ge innerhalb der Liga (zum Beispiel jeder Spieler  spielt jede zweite Woche ein Spiel)
\end{itemize}
\item Android App, welche auf die WebAPI zugreift mit gleicher Funktionalit�t wie Web Applikation
\end{itemize}