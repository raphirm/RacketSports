\chapter{Definition Aussp�hen}
\section{Definition}
Aussp�hen bedeutet Informationen �ber ein Ziel sammeln. 

In der IT ist Aussp�hen eine Voraussetzung um Informationen zu entwenden und oder einen Angriff gegen ein IT-System zu starten. Aussp�hen kann auch benutzt werden, um den Urheber hinter einem IT-System zu finden. 

Die IT-Sicherheit definiert Aussp�hen in dem CIA Paradigma \footcite{cia} unter dem Begriff "Confidentiality" (auf deutsch Vertraulichkeit). 

In dieser Arbeit wird grob in drei verschiedene Arten unterschieden. Diese Arten unterscheiden sich in der Aggressivit�t, um an Informationen zu kommen. Sp�ter im Dokument werden alle Techniken in diese Arten aufgegliedert. Viele Techniken k�nnen nicht eindeutig auf eine Art zugeordnet werden und bestehen je nach Anwendung aus einer Mischung von verschiedenen "'Aggressivit�t Level"'.

\subsection{Footprinting}
Footprinting beschreibt Technologien, welche unauff�llig oder teilweise ohne Kontakt zum Ziel agieren. Diese Technologien greifen zum Beispiel auf �ffentliche Quellen zu oder pr�fen nur die Existenz oder den Weg zum Ziel. Footprinting sammelt generelle Informationen des Ziels, die f�r irgendwelche (nicht zwangsl�ufig kriminelle) Zwecke weiterverwendet werden k�nnen.

\subsection{Vulnerability Scanning}
Vulnerability Scanning sp�ht spezifisch Schwachstellen (engl. Vulnerabilities) aus. Diese Informationen k�nnen benutzt werden, um Zugang zu fremden Systemen oder Firmen zu bekommen. Bei Vulnerability Scanning werden meist Versionen von Programmen und Systemen aufgerufen und aufgrund dessen Informationen bekannte Schwachstellen zugeordnet. Bei Vulnerability Scans wird meist die Schwachstelle nicht ausgenutzt, sondern nur Informationen dar�ber gesammelt. 

\subsection{Penetration}
Penetration Testing geht einen Schritt weiter als Vulnerability Scanning. Neben bekannten Schwachstellen werden auch sogenannte "'generische"' Schwachstellen (z.B. Cross Site Scripting oder SQL Injuctions) gesucht und auch missbraucht, um herauszufinden, was f�r Auswirkungen die Schwachstelle auf das System hat und welche Rechteeskalation anschliessend m�glich ist.

Als Beispiel f�r den Unterschied zwischen Penetration Testing und Vulnerability Scanning: Benutzt eine Webseite ein Content Management System (z.B. Joomla),  so sucht der Vulnerability Scan die Version heraus und benutzt Datenbanken, die Schwachstellen protokollieren, um herauszufinden welche Schwachstellen f�r die jeweilige Version vorhanden ist. Der Penetration Test im Gegensatz versucht z.B. die Tabellen im Hintergrund mit speziellen SQL Befehlen �ber GET und POSTS zu hacken oder mit Cross Site Scripts den Prozess hinter der Webseite zu hacken.

\section{Ziele}
\subsection{Personen und Unternehmen}
In diesem Dokument wird zwischen zwei verschiedenen Zielen unterschieden. Wenn das Ziel Personen und Unternehmen sind, geht es prim�r nicht darum, eine IT-Infrastruktur auszusp�hen, sondern mithilfe von IT-Hilfsmitteln eine Person auszuspionieren. Es werden meistens logische Schwachstellen im Internet benutzt, um mehr �ber eine bestimmte Person oder ein bestimmtes Unternehmen zu erfahren. (Ein perfektes Beispiel war vor ein paar Jahren die Zugriffsberechtigungen von Facebook, mit welchen man das ganze Beziehungsnetz inklusive Fotos und weiteren Informationen von unbekannten aussp�hen konnte.)

\subsection{IT-Systeme}
Wenn man ein IT-System als Ziel hat, ist die Person dahinter meistens nicht im Mittelpunkt. (Mindestens f�r die unmittelbare Methode, f�r einen ganzheitlichen Angriff ist die Person m�glicherweise dennoch interessant.) Bei einer Informationsbeschaffung von IT-Systemen versucht man herauszufinden, was das System freiwillig von sich preis gibt. Die Erfahrung zeigt, dass IT-Systeme gerne viel mitteilen. Auch andere unbeteiligte Systeme k�nnen technische Informationen des Zielsystems enthalten. 

\section{Techniken}
\subsection{Passives Aussp�hen}
Passives Aussp�hen heisst, der Spion hat keinen direkten Kontakt mit dem Ziel. Der Vorteil von passivem Aussp�hen ist, dass das Ziel nahezu keine Chance hat ein solcher Spionageversuch aufzudecken oder zu detektieren. 

\subsection{Aktives Aussp�hen}
Aktives Aussp�hen ist das genaue Gegenteil. Hier interagiert man mit dem Ziel. Das Risiko entdeckt zu werden ist bei aktivem Aussp�hen erheblich gr�sser, daf�r kann man sehr viel mehr Informationen erhalten.

\subsection{Beispiel zur Unterscheidung}
Auf den ersten Blick ist der Unterschied von aktivem und passivem Aussp�hen schwer ersichtlich. Darum gibt es hier ein einfaches Beispiel:
Ausgangslage: Man will �ber sein Ziel wissen, welche Produkte das Ziel �ber die Homepage vertreibt.  
Aktives Aussp�hen: Man besucht die Homepage d.h. man interagiert mit dem Webserver des Ziels. 
Passives Aussp�hen: Man benutzt den Cache von Google und interagiert mit Google anstatt mit dem Ziel.